\bfsection{\S 6 素因子と準素分解}

\bfsubsection{\S 6 冒頭}
\barquo{
$A$の素イデアル$P$が$M$の適当な元$x$のannihilatorとなるとき、$P$を$M$の素因子という。
}
\begin{rem}
この定義は、$A/P \hookrightarrow M$が成り立つことと同値である。証明はやさしいので各自試みよ。
\end{rem}



\bfsubsection{定理 6.2}
\barquo{
$\Spec (A_S)$を$\Spec(A)$の部分集合とみなすとき、$\Ass_A(N)= \Ass_{A_S}(N)$である。$A$がネーター環なら、$A$加群$M$に対し$\Ass(M_S)=\Ass(M) \cap \Spec(A_S)$が成り立つ。
}
\begin{rem}
  本文を読んでも、何を主張していてどこでネーター性を使っているのか分かりにくい。次のようにいくつかの補題に分けて証明する。
\end{rem}

\lem{
$N \in \Mod{A_S}$に対して、次は可換。
\[
\xymatrix{
N \ar[r]^{id} \ar[d]_{\ann} & N \ar[d]^{\ann} \\
P(A_S) \ar[r]^b & P(A)
}
\]
すなわち、任意の$s \in N$に対して
\[
\ann_A(x) = \ann_{A_S}(x) \cap A
\]
が成り立つ。
}

\begin{proof}
$a \in A$とする。このとき
\begin{align*}
  a \in \ann_A{x} &\iff ax=0 \\
  &\iff \f{a}{1} \cdot x = 0 \\
  &\iff \f{a}{1} \in \ann_{A_S}(x) \\
  &\iff a \in \ann_{A_S}(s) \cap A
\end{align*}
が成り立つ。
\end{proof}

\lem{
  $\Spec(A \setminus S)$で$\setmid{\frakp \in \Spec(A)}{\frakp \cap S = \emptyset}$を表すことにする。このとき次は可換。
  \[
  \xymatrix{
  \Mod{A_S} \ar[r] \ar[dd]_{\Ass} & \Mod{A} \ar[d]^{\Ass} \\
  {} & P(\Spec (A)) \\
  P(\Spec (A_S)) \ar[r]^{b_*} & P(\Spec (A \setminus S)) \ar[u]_{i_*}
  }
  \]
  すなわち、任意の$N \in \Mod{A_S}$に対して
  \[
  \Ass_A(N) = \Ass_{A_S}(N)
  \]
  が成り立つ。
}

\begin{proof}
  任意の$\frakp \in \Spec(A)$について
  \begin{align*}
    \frakp \in \Ass_A(N) &\iff \exists x \in N \st \frakp = \ann_A(x) \\
    &\iff \frakp = \ann_A(x)  \; \text{かつ} \; \frakp \in \Spec(A \setminus S) \\
    &\iff \frakp = \ann_{A_S}(x) \cap A \; \text{かつ} \; \frakp \in \Spec(A \setminus S) \\
    &\iff \frakp A_S \in \Ass_{A_S}(N) \\
    &\iff \frakp \in \Ass_{A_S}(N)
  \end{align*}
  が成り立つことから判る。
\end{proof}

\lem{
    $A$加群$M$について次は可換。
    \[
    \xymatrix{
    M_S \ar[d]_{\ann} & M \ar[l] \ar[d]^{\ann} \\
    P(A_S) & P(A) \ar[l]^g
    }
    \]
    すなわち$x \in M$について
    \[
    \ann_A(x) A_S = \ann_{A_S}(x/1)
    \]
}

\begin{proof}
  $a/s \in A_S$に対して
  \begin{align*}
    \f{a}{s} \in \ann_A(x) A_S &\iff \exists t \in S \st ta \in \ann_A(x) \\
    &\iff tax = 0 \\
    &\iff ax = 0 \in M_S \\
    &\iff a \in \ann_{A_S}(x)
  \end{align*}
\end{proof}





\lem{
$A$がNoether環、$M$が$A$加群で、$x \in M$なら
\[
\ann_{A_S}(x/1) \cap A = \ann_A(tx)
\]
なる$t \in S$がある。
}

\begin{proof}
  $\ann_{A_S}(x/1) \cap A = \ann_A(x/1)$なので、$\ann_A(x/1) = \ann_A(tx)$なる$t \in S$があることを示せば十分である。

  $a \in \ann_A(x/1)$とする。このとき$a \cdot \f{x}{1} = 0 \; \in  M_S$であるから、ある$s \in S$が存在して$sax=0 \; \in M$が成り立つ。ゆえに$a \in \ann_A(sx)$である。$a$の選び方により$s$は変わるので$\ann_A(x/1) \subset \bigcup_{s \in S} \ann_A(sx)$である。逆は明らかだから、
  \[
  \ann_A(x/1) = \bigcup_{s \in S} \ann_A(sx)
  \]
  である。$A$はNoetherなので集合族$\setmid{\ann_A(sx)}{s \in S}$には極大元$\ann_A(tx)$が存在する。すると任意の$s \in S$について、$S$が積閉集合であることと極大性により$\ann_A(tx) = \ann_A(stx) \supset \ann_A(sx)$が判る。したがって$\ann_A(tx)$は実は最大元であり、
  \[
  \ann_A(x/1) = \ann_A(tx)
  \]
  が成り立つ。
\end{proof}


\lem{
    $A$がNoether環なら、次は可換。
    \[
    \xymatrix{
    \Mod{A_S} \ar[dd]_{\Ass} & \Mod{A} \ar[l] \ar[d]^{\Ass}  \\
    {} & P(\Spec (A)) \ar[d]^{i^*} \\
    P(\Spec (A_S)) \ar[r]^{b_*} & P(\Spec (A \setminus S))
    }
    \]
    すなわち、$A$加群$M$について
    \[
    \Ass_{A_S}(M_S) = \Ass_A(M) \cap \Spec(A_S)
    \]
    が成り立つ。
}

\begin{proof}
  $\frakp \in \Spec (A \setminus S)$とする。このとき
  \begin{align*}
    \frakp \in \Ass_A(M) &\iff \exists x \in M \st \frakp = \ann_A(x) \\
    &\iff \exists s \in S \st \frakp = \ann_A(s \cdot x) \\
    &\iff \frakp A_S = \ann_{A_S}(x/1) \\
    &\iff \frakp A_S \in \Ass_{A_S}(M_S)
  \end{align*}
  であるから、示したいことが言えた。
\end{proof}


\bfsubsection{定理 6.3}
\barquo{
$\Ass(M) \subset \Ass(M') \cup \Ass(M'')$
}
\begin{rem}
  等号が成り立たない例としては次の完全系列がある。
  \[
  \xymatrix{
  0 \ar[r] & \Z \ar[r]^{\times 12} & \Z \ar[r] & \Z / 12 \Z \ar[r] & 0
  }
  \]
  このとき$\Ass(\Z) = \{ (0) \} $であるが$\Ass( \Z / 12 \Z) = \{ (2), (3)\}$であるので等号が成立しない。
\end{rem}


\bfsubsection{定理 6.5}
\barquo{
$\Ass(A/P)=\{ P \}$であることに注意すればよい。
}
\begin{proof}
$\{ P \} \subset \Ass(A/P)$は、$P=\ann(1)$よりあきらか。逆に$Q \in \Ass(A/P)$であるとする。このとき素因子の定義から$Q = \ann(\ol{x})$なる$\ol{x} \in A/P$がある。したがって$P \subset Q$である。一方で$y \in Q$とすると$yx \in P$が従う。$x$は$P$の元ではあり得ないので、$P$が素イデアルであることより$y \in P$である。よって$P \supset Q$
、すなわち$P = Q$である。
\end{proof}





\bfsubsection{定理 6.5}
\barquo{
iii) $P$を$\Supp(M)$の極小元として$P \in \Ass(M)$を示せばよい。
}
\begin{rem}
  $\Ass(M)$における極小元が$\Supp(M)$でも極小元であることも確かめる必要がある。

  いま$P \in \Ass(M)$が$\Ass(M)$において極小であるとする。任意に$Q \subset P$なる$Q \in \Supp(M)$が与えられたとしよう。このとき$M_Q \neq 0$であり、$A$はNoetherなので$\Ass_{A_Q}(M_Q) \neq \emptyset$である。ゆえに
  \begin{align*}
  \emptyset \neq \Ass_{A_Q}(M_Q) &= \Ass_A(M) \cap \Spec (A_Q) \\
  &\subset \Ass_A(M) \cap \Spec (A_P) \\
  &= \{ P \}
\end{align*}
が成り立つので$P \in \Ass_{A_Q}(M_Q) \subset \Spec (A_Q)$でなくてはならない。よって$P = Q$である。$Q$は任意だったので、$P$は$\Supp(M)$でも極小ということが示せた。
\end{rem}




\bfsubsection{定理 6.5 直後}
\barquo{
$A$をNoether環、$M$を有限$A$加群とする。$\Supp(M)$の極小元を$P_1, \cdots , P_r$とすれば$\Supp(M)= V(P_1) \cup \cdots \cup V(P_r)$であり
}
\begin{proof}
  $A$がNoether環で$M$が有限$A$加群という仮定から、$\Supp(M)$の極小元は有限個しかないので、$\Supp(M)$の極小元を$P_1, \cdots , P_r$とおくことができる。このとき$\Supp(M) \subset V(P_1) \cup \cdots \cup V(P_r)$である。逆に、$Q \in V(P_1) \cup \cdots \cup V(P_r)$なら$Q \supset P_i$なる$i$がある。すると
  \[
  0 \neq M_{P_i} = M_Q \otimes_A A_{P_i}
  \]
  より$M_Q \neq 0$なので$Q \in \Supp(M)$である。
\end{proof}




\bfsubsection{定理 6.6}
\barquo{
前定理により$\Supp (M/N) = V(\frakp)$、したがって$P = \sqrt{\ann(M/N)}$である。
}
\begin{proof}
  $M$は有限生成なので、$M/N$も有限生成。よって
  \[
  \Supp (M/N) = V(\ann(M/N))
  \]
  が成り立つ。したがって
  \begin{align*}
    \sqrt{\ann(M/N)} &= \bigcap_{\frakq \in V(\ann(M/N))} \frakq \\
    &= \bigcap_{\frakq \in V(\frakp)} \frakq \\
    &= \sqrt{\frakp} \\
    &= \frakp
  \end{align*}
 が判る。
\end{proof}





\bfsubsection{定理 6.8 (i)}
\barquo{
$x \in K_i$, $x \neq 0$なら$\ann(x)=P_i$であるから
}
\begin{rem}
  この部分は定理6.3の証明と同様。
\end{rem}



\bfsubsection{定理 6.8 (ii)}
\barquo{
一方$N : M$は$P_1$に属する準素イデアルだから
}
\begin{rem}
  要するに$\sqrt{\ann(M/N)} = P_1$ということである。
\end{rem}



\bfsubsection{定理 6.8 (iii)}
\barquo{
$i > 1$に対し$\ann(M/N_i)$は$P_i$のべきを含み、$P_i \not\subset P_1$であるから$(M/N_i)_P=0$であり、
}
\begin{proof}
  $P_i$は$\ann(M/N_i)$の根基で、$A$はNoetherなので$P_i^{v} \subset \ann(M/N_i)$なる$v \geq 1$がある。$P = P_1$は極小素因子としてとってきたので$P_i \not\subset P_1$であり、$P$は素イデアルなので$P_i^v \cap P^c \neq \emptyset$である。よって$\ann(M/N_i) \cap P^c \neq \emptyset$である。したがって$(M/N_i)_P=0$がわかる。
\end{proof}




\bfsubsection{定理 6.8 (iii)}
\barquo{
したがって$\vp_P^{-1}(N_P) = \vp_P^{-1}((N_1)_P)$であり、この右辺は容易に確かめられるように$N_1$に等しい。
}
\begin{proof}
  $\vp_P^{-1}((N_1)_P) = N_1$を示そう。$\vp_P^{-1}((N_1)_P) \supset N_1$はあきらかなので逆を示す。

  $x \in \vp_P^{-1}((N_1)_P)$とする。このとき$x/1 \in (N_1)_P$であるから、ある$s \in A \setminus P$が存在して$s x \in N_1$である。このときもし$x \not\in N_1$ならば、$s$は$M/N_1$-非正則ということになり、$s \in P$でなければならない。これは矛盾。
\end{proof}
