\bfsection{\S 10 一般付値}

\bfsubsection{\S 10 冒頭}
\barquo{
整域$R$が付値環であるとは、その商体$K$の各元$x$について
\[
x \not\in R \To x^{-1} \in R
\]
が成り立つことをいう。
}
\begin{rem}
  付値環の例として、たとえばべき級数環$R = \Q[[x]]$がある。これが付値環であることの証明は省略する。
\end{rem}


\bfsubsection{定理 10.2}
\barquo{
$A$を$A_{\frakp}$でおきかえて、$A$が局所環で$\frakp = \frakm_A$としてよい。
}
\begin{proof}
  書いてしまえばほとんど当たり前だが、一応説明しておく。局所環の場合に示せたとする。このとき、$A_{\frakp}$は局所環なので
  \[
  R \supset A_{\frakp}, \quad \frakm_{R} \cap A_{\frakp} = \frakp A_{\frakp}
  \]
  なる$K$の付値環$R$が存在する。このとき$R \supset A$であり、かつ
  \[
  \frakm_{R} \cap A = (\frakm_{R} \cap A_{\frakp}) \cap A = \frakp A_{\frakp} \cap A = \frakp
  \]
  が成り立つ。よって$A$について示すべきことがいえた。
\end{proof}




\bfsubsection{定理 10.4}
\barquo{
$K$の付値環で$A$を含むものたちの共通部分を$B'$とすれば、前定理により$B \subset B'$である。
}
\begin{proof}
  $x \in B$とし、$K$の付値環$R$で$A$を含むものがあたえられたとする。このとき$x$は$A$上整なので$R$上でも整である。したがって、付値環は整閉なので$x \in R$である。よって$B \subset R$であり、$R$は任意だったから$B \subset B'$である。
\end{proof}


\bfsubsection{定理 10.5}
\barquo{
すると$\scra$は次の性質をもつことが容易にわかる。
\begin{description}
  \item[$\gra$)] $F_1 ,\cdots , F_{r} \in \scra \; \Rightarrow \; F_1 \cap \cdots \cap F_{r} \in \scra$
  \item[$\beta$)] $Z_1 , \cdots , Z_n$が閉集合で$Z_1 \cup \cdots \cup Z_n \in \scra \; \Rightarrow \; \exists i : Z_i \in \scra$
  \item[$\grg$)] 閉集合$F$が$\scra$の元を含めば$F \in \scra$
\end{description}
}
\begin{proof} ${}$
  \begin{description}
    \item[$\gra$)] $\scra \cup \{F_1 \cap \cdots \cap F_{r}\}$は有限交性をもつので、$\scra$の極大性から。
    \item[$\beta$)] 有限個の閉集合$Z_1 , \cdots , Z_n$が与えられたとし、任意の$i$について$Z_i \not\in \scra$だとする。$\scra$の極大性により、$\scra \cup \{Z_i \}$は有限交性をもたない。したがって$\gra$)により$Z_i \cap F_i = \emptyset $なる$F_i \in \scra$が$i$ごとに存在する。このとき$(Z_1 \cup \cdots \cup Z_n) \cap \bigcap_{i=1}^n F_i = \emptyset$
    である。ゆえに有限交性がなくなるので$Z_1 \cup \cdots \cup Z_n \not\in \scra$である。
    \item[$\grg$)] $\scra \cup \{ F \}$は有限交性をもつので、$\scra$の極大性から。
  \end{description}
\end{proof}




\bfsubsection{定理 10.5}
\barquo{
補集合を表すには肩に$c$をつけることにすると、$F \in \scra$, $F^c = \bigcup_{\grl} U_{\grl}$なら$F = \bigcap_{\grl} U_{\grl}^c$であり、また$U(x_1, \cdots , x_n)^c = \bigcap_{i=1}^n U(x_i)^c \in \scra$なら上の$\beta)$によりどれかの$U(x_i)^c$が$\scra$に属する。よって$\scra$の元の共通部分は$\scra$に属する$U(x)^c$の形の集合の共通部分に等しい。
\[
\grG = \setmid{y \in K}{U(y^{-1})^c \in \scra }
\]
とおく。$R \in \Zar (K,A)$が$U(y^{-1})^c$に属するための条件は$y \in \frakm_R$であるから
\[
\text{$\scra$のすべての元の共通部分} = \setmid{R \in \Zar(K,A)}{\frakm_R \supset \grG}
\]
である。
}
\begin{rem}
  いろいろ書いてあるが、証明で必要なのは
  \[
  \text{$\scra$のすべての元の共通部分} \supset \setmid{R \in \Zar(K,A)}{\frakm_R \supset \grG}
  \]
  だけである。他は無視してもよい。そこで、これだけを示すことにする。簡単に示せる次の補題に気をつける。
\end{rem}

\lem{
$\grd$) \\
$Z_{\grl} \; (\grl \in \grL)$が閉部分集合で$\bigcap_{\grl} Z_{\grl} \in \scra$ならば、すべての$\grl \in \grL$について$Z_{\grl} \in \scra$である。
}
\begin{proof}
  $\grg$)によりあきらか。
\end{proof}

\begin{proof}
  引用部の証明に戻る。$\frakm_R \supset \grG$なる付値環$R \in \Zar(K,A)$と$F \in \scra$が任意に与えられたとする。補集合は$\Zar(K,A)$を全体集合としてとると約束しよう。このときZariski位相の定義により
  \[
  F^c = \bigcup_{\grl \in \grL} U_{\grl}
  \]
  なる開基$\scrf$の元$U_{\grl} $がある。このとき$\grl$ごとに有限集合$B(\grl)$が存在して
  \[
  U_{\grl}^c = \bigcup_{\tau \in B(\grl)} U(x_{\tau})^c
  \]
  が成り立つ。したがって
  \begin{align*}
    F &= \bigcap_{\grl \in \grL} U_{\grl}^c \\
    &= \bigcap_{\grl \in \grL} \bigcup_{\tau \in B(\grl)} U(x_{\tau})^c
  \end{align*}
  が成り立つ。$F \in \scra$と補題$\grd$)により、任意の$\grl$に対して
  \[
  \bigcup_{\tau \in B(\grl)} U(x_{\tau})^c \in \scra
  \]
  である。$\beta$)により、$\grl$ごとにある$\tau(\grl) \in B(\grl)$が存在して$U(x_{\tau(\grl)})^c \in \scra$である。すると
  \[
  F \supset \bigcap_{\grl \in \grL} U(x_{\tau(\grl)})^c
  \]
  となる。ここで$x_{\tau(\grl)}^{-1} \in \grG$より仮定から$x_{\tau(\grl)}^{-1} \in \frakm_R$となる。よって$x_{\tau(\grl)} \not\in R$なので$R \in U(x_{\tau(\grl)})^c $が結論される。$\grl$は任意だったから
  \[
  R \in  \bigcap_{\grl \in \grL} U(x_{\tau(\grl)})^c \subset F
  \]
  であって、これで示すべきことがいえた。
\end{proof}


\bfsubsection{定理 10.5 直後}
\barquo{
$G$は包含関係で全順序集合であるが、われわれは$G$に包含関係と逆の順序を入れることにする。
}
\begin{rem}
  包含関係と同じ順序をいれたとしても、順序群(あとで定義する)にはなっている。
\end{rem}


\bfsubsection{定理 10.5 直後}
\barquo{
公理から、
\[
(1) \; x>0,y \geq 0 \Rightarrow x+y > 0, \quad (2) \; x \geq y \Rightarrow -y \geq -x
\]
などが従う。
}
\begin{proof} ${}$
  \begin{description}
    \item[(1)] $x \geq 0$より$x + y \geq 0$である。もしも$x+y=0$なら、$x=-y$である。$-y \geq -y$と$y \geq 0$により$0 \geq -y$だから、$0 \geq x$となり、矛盾。よって$x+y>0$である。
    \item[(2)] $-x-y \geq -x-y$だから、両辺足して$-y \geq -x$を得る。
  \end{description}
\end{proof}


\bfsubsection{定理 10.5 直後}
\barquo{
$R_{\nu}$は$K$の付値環で$m_{\nu}$がその極大イデアルである。
}
\begin{proof}
  $R_{\nu}$が$K$の部分環 ($1,0 \in R$かつ$x,y \in R$なら$xy,x-y \in R$) であり、その商体が$K$で、付値環であるということ、そして$m_{\nu} \subset R_{\nu}$が全体ではないイデアルであるということはあきらか。極大イデアルであることを示そう。いま$x \in R_{\nu} \setminus m_{\nu}$とする。
  \[
  0 =\nu(1) = \nu (x x^{-1}) = \nu(x) + \nu(x^{-1}) = \nu(x^{-1})
  \]
  より$x^{-1} \in R_{\nu}$である。よって示せた。
\end{proof}



\bfsubsection{定理 10.5 直後}
\barquo{
対応$\nu \colon K \to G \cup \{\infty \}$を$\nu(0)=\infty$, $\nu(x)=xR \; (x \in K^*)$で定義すれば、$\nu$は$G$を値群とする加法付値になり
}
\begin{proof}
  加法付値の条件(1),(2),(3)を確認しよう。といっても(1)と(3)はあきらかなので、(2)だけを示す。$x,y,x+y$のいずれかが$0$なら証明することはないので、これらは$K^*$の元としてよい。$\nu(x) \geq \nu(y)$と仮定しよう。$xR \geq yR$より、順序の入れ方から$xR \subset yR$である。したがって$x = ry$なる$r \in R$がある。ゆえに$(x+y)R \subset yR$だから$\nu(x+y) \geq \nu(y)$である。よって示すべきことがいえた。
\end{proof}



\bfsubsection{定理 10.5 直前}
\barquo{
$H,H'$を値群とする$K$の二つの加法付値$\nu, \nu'$が共に$R$を付値環としてもてば、$H$から$H'$の上への順序を保つ同形写像$\vp$があって$\nu' = \vp \nu$が成り立つ(証明してみよ)。
}
\begin{proof}
  準同形定理により$K^* / R^{\tm} \cong H$かつ$K^* / R^{\tm} \cong H'$である。したがって次の図式
  \[
  \xymatrix{
  K^*/ R^{\tm} \ar[r]^{\ol{\nu} } \ar[dr]_{\ol{\nu'}} & H \ar[d]^{\vp} \\
  {} & H'
  }
  \]
を可換にするような群の同型$\vp$がある。$\vp$が順序を保つことをみよう。$x \in H$, $x \geq 0$とする。このとき$x = \nu(b)$なる$b \in R$がある。したがって
\begin{align*}
  \vp(x) &= \vp(\ol{\nu}(\ol{b}) ) \\
  &= \ol{\nu'}(\ol{b}) \\
  &= \nu'(b) \\
  &\geq 0
\end{align*}
である。
  \begin{comment}
  証明してみた。次の図式を可換にする$\vp$をみつければよい。
  \[
  \xymatrix{
  K^* \ar[r]^{\nu} \ar[dr]_{\nu'} & H \ar[d]^{\vp} \\
  {} & H'
  }
  \]
  $x \in H$とすると$x=\nu(a)$なる$a \in K^*$があるので、このとき
  \[
  \vp(x)=\nu'(a)
  \]
  と定めれば(well-definedならば)可換にできる。well-definedであることを示そう。$x = \nu(a) = \nu(b)$なる$a,b \in K^*$があったとする。このとき$ab^{-1} \in \Ker \nu$である。$R$は付値環なので、$ab^{-1}$と$a^{-1}b$のどちらかは$R$の元であり、したがって$ab^{-1}$は$R$の単元である。ゆえに$\nu'(ab^{-1})=0$であって、$\nu'(a)= \nu'(b)$となる。よって$\vp$はwell-defindである。

  $x = \nu(a)$かつ$y = \nu(b)$だとする。このとき
  \begin{align*}
    \vp(x+y) &= \vp(\nu(a)+\nu(b) ) \\
    &= \vp(\nu(ab)) \\
    &= \nu'(ab) \\
    &= \nu'(a) + \nu'(b) \\
    &= \vp(x) + \vp(y)
  \end{align*}
  だから$\vp$は準同形。

  もし$x \geq 0$ならば、$x = \nu(a)$なる$a$は$R$の元である。よって$\vp(x)=\nu'(a) \geq 0$である。つまり$\vp$は順序を保つ。
\end{comment}
\end{proof}



\bfsubsection{定理 10.6}
\barquo{
任意の$y \in G$に対し、$nx \leq y$となる$n \in \Z$の中で最大のものが定まる。
}
\begin{proof}
  仮定により$n' x > y$なる$n'$があり、$n'$より大きい$n$についても$nx>y$なので、集合$B(y) = \setmid{n \in \Z}{nx \leq y}$は上に有界である。集合$B(y)$が空でないことは、$y \geq 0$のときは$0 \in B(y)$よりわかる。よって、$y \geq 0$ならば$B(y)$は最大元を持つ。$y < 0$とする。このとき、$B(-y)$は最大元を持つので、
  \[
  nx \leq -y < (n+1)x
  \]
  となるような$n$がある。このとき$-(n+1) \in B(y)$である。したがって空でないので、$B(y) \; (y<0)$も最大元を持つ。
\end{proof}


\bfsubsection{定理 10.6}
\barquo{
$y_1 = y - n_0 x$とおき$nx \leq 10y_1$となる最大の整数$n$を$n_1$とおけば$0 \leq n_1 < 10$である。
}
\begin{proof}
  $0 \leq y_1$より、$n_1$の最大性から$0 \leq n_1$がわかる。また$10$を代入してみると
  \begin{align*}
    10y_1 - 10x &= 10y - 10n_0x - 10 x \\
    &= 10y - 10(n_0 + 1)x \\
    &< 0 &(n_0\text{の最大性から})
  \end{align*}
  が成り立つため、$n_1 < 10$である。
\end{proof}



\bfsubsection{定理 10.6}
\barquo{
$\vp \colon G \to \R$が順序集合として準同形、すなわち$y < y'$ならば$\vp(y) \leq \vp(y')$であることが容易にたしかめられる。
}
\begin{proof}
  $y < z$と仮定する。
  \begin{gather*}
    n_0 = \max\setmid{n}{nx \leq y} \\
    m_0 = \max\setmid{m}{mx \leq z}
  \end{gather*}
  とすると$n_0 \leq m_0$である。$n_0 < m_0$なら示すべきことはないので$n_0 = m_0$とする。このとき$y_1 = y - n_0x$, $z_1 =z - m_0 x $とすると$y_1 < z_1$であって、
  \begin{gather*}
    n_1 = \max\setmid{n}{nx \leq 10y_1} \\
    m_1 = \max\setmid{m}{mx \leq 10z_1}
  \end{gather*}
  とすると$n_1 \leq m_1$である。$n_1 < m_1$なら示すべきことはないので$n_1 = m_1$とする。このとき$y_2 = 10y_1 - n_1x$, $z_2 =z_1 - m_1 x $とすると$y_2 < z_2$である。以下、繰り返しにより$n_i$, $m_i$はすべて一致するか、またはある$k$について$n_k < m_k$かつ、すべての$i<k$について$n_k = m_k$である。よって$\vp(y) \leq \vp(z)$がいえた。
\end{proof}


\bfsubsection{定理 10.6}
\barquo{
さらに$\vp$は単射である。それを見るには、$y < y'$なら$x < 10^r(y' - y)$となる自然数$r$が存在することを注意すればよい。(詳細は読者に委ねる)
}
\begin{rem}
  委ねられてしまった。次の補題を使う。
\end{rem}

\lem{
定理10.6の状況で考える。$r \geq 0$に対して
\[
N(r) = \sum_{i=0}^r 10^{r-i}n_i
\]
とおく。このとき
\[
N(r)x = 10^ry -y_{r+1}
\]
が成り立つ。
}
\begin{proof}
  証明は$r \geq 0$についての帰納法による。$r=0$のとき$N(r)=n_0$であり、$n_0x=y-y_1$だから成り立っている。$r \geq 1$とし、$r-1$までの成立を仮定する。このとき
  \begin{align*}
    N(r)x &= (10N(r-1) + n_r)x \\
    &= 10N(r-1)x + n_rx \\
    &= 10(10^{r-1}y -y_r) + n_rx &(\text{帰納法の仮定による}) \\
    &= 10^r y - 10 y_r + n_r x \\
    &= 10^r y - y_{r+1} &(r \geq 1\text{より})
  \end{align*}
  だから、示すべきことがいえた。
\end{proof}

\lem{
$r \geq 1$のとき$0 \leq  y_r < x $が成り立つ。
}
\begin{proof}
  $n_0 x \leq y < (n_0 + 1)x$により$0 \leq y - n_0 x < x$である。よって$0 \leq y_1 < x$が成りたつ。また$k \geq 1$のとき$n_k x \leq 10y_k < (n_k + 1)x$であるので、$0 \leq 10y_k - n_kx  < x$がわかる。したがって$0 \leq y_{k+1} < x$である。
\end{proof}

\lem{
$n \in \Z$, $r \geq 0$, $y \in G$とする。次は同値。
\begin{description}
  \item[(1)] $n = N(r)$である。とくに$\vp(y)$の小数第$r$桁までの表示であって、$\f{n}{10^r}$と等しいものがある。(一般に小数展開は一意ではないことに注意)
  \item[(2)] $nx \leq 10^r y < (n+1)x$が成り立つ。
\end{description}
}
\begin{proof}
  ${}$
  \begin{description}
    \item[(1)$\To$(2)] $10^r y - x < nx \leq 10^r y$を示せばよいが、$0 \leq y_{r+1} < x$と$N(r)x = 10^r y - y_{r+1}$であることからそれはあきらか。
    \item[(2)$\To$(1)] (2)の両辺から$y_{r+1}$を引いて
    \[
    nx - y_{r+1} \leq N(r)x < (n+1)x - y_{r+1}
    \]
    である。よって
    \[
    (n - N(r))x \leq y_{r+1} < (n+1 - N(r))x
    \]
    がわかる。ここで$0 \leq y_{r+1} < x$により、$-1 < n - N(r) < 1$でなくてはならない。$n - N(r) \in \Z$だから、$n=N(r)$でしかありえない。
  \end{description}
\end{proof}

\begin{proof}
引用部の証明に戻る。$\vp(y) = \vp(y')$であったとしよう。このときすべての$r \geq 0$に対して$N(r) = N'(r)$であるので、
\begin{align*}
  N(r)x \leq &10^r y < (N(r)+1)x \\
  N(r)x \leq &10^r y' < (N(r)+1)x
\end{align*}
である。したがって
\begin{align*}
  &10^r(y - y') \leq (N(r)+1)x - N(r)x = x \\
  &10^r(y' - y) \leq (N(r)+1)x - N(r)x = x
\end{align*}
が成り立つ。ここでもしも$y \neq y'$ならば、仮定により十分大きな$r$についてどちらかは成り立たないはずなので、矛盾。よって$y = y'$である。
\end{proof}


\bfsubsection{定理 10.6}
\barquo{
次に$\vp$が群の準同形であることを見よう。
}
\begin{proof}
  $y,y' \in G$をとったとする。$N'(r)$を$N(r)$と同様のものとする。このとき
  \begin{align*}
N(r)x \leq 10^r y < (N(r)+1)x \\
N'(r)x \leq 10^r y < (N'(r)+1)x
  \end{align*}
  である。両辺足して
  \[
(N(r)+N'(r))x \leq 10^r y < (N(r) + N'(r) + 2)x
  \]
  となる。したがって$M(r)=N(r)+N'(r)$とおくと、
  \[
  M(r)x \leq 10^r(y+y') < (M(r)+1)x
  \]
  かまたは
  \[
  (M(r)+1)x \leq 10^r(y+y') < (M(r)+2)x
  \]
  のどちらかが成り立つ。したがって、
  \[
  0 \leq \vp(y+y') - \f{M(r)}{10^r} < \f{1}{10^r}
  \]
  \[
  0 \leq \vp(y+y') - \f{M(r)+1}{10^r} < \f{1}{10^r}
  \]
  のどちらかは正しい。このときいずれにせよ
  \[
  0 \leq \vp(y+y') - \f{M(r)}{10^r} < \f{2}{10^r}
  \]
  である。ゆえに
  \begin{align*}
    \abs{\vp(y+y') - \vp(y) - \vp(y')} &\leq \abs{\vp(y+y') - \f{M(r)}{10^r}} + \abs{ \f{N(r)}{10^r} - \vp(y) } + \abs{ \f{N'(r)}{10^r} - \vp(y') } \\
    &\leq \f{2}{10^r} + \f{1}{10^r} + \f{1}{10^r} \\
    &\leq \f{4}{10^r}
  \end{align*}
  がわかる。$r$は任意だから、$\vp(y+y') = \vp(y) + \vp(y')$がいえた。
\end{proof}


\bfsubsection{定理 10.7}
\barquo{
付値環$R$の値群を$G$とすると
\[
G\text{が階数$1$} \iff R\text{のKrull次元が$1$}
\]
}
\begin{proof}
  $\To$は本文通り。$\Leftarrow$を示す。$\dim R = 1$より、$R$は体ではない。したがって$G = \nu(R \setminus \{0\} )$は$0$ではない。ゆえに、定理10.6により、$a,b \in G$であって$a>0$なるものが与えられたとして、$na > b$なる$n$が存在することをいえばよい。$b \leq 0$ならば示すことはないので、$b>0$としてよい。

  $\nu$を$R$に対応する加法付値とする。$a=\nu(\xi)$, $b=\nu(\eta)$を満たす$\eta , \xi \in \frakm_R \setminus \{0\}$がある。$\dim R = 1$であるという仮定と、$R$のイデアル全体は全順序集合をなすということから、$\frakm_R$は$\eta \frakm_R$を含む唯一の素イデアルである。ゆえに
  \[
  \frakm_R = \sqrt{\eta \frakm_R}
  \]
  であるので、$\xi^n \in \eta \frakm_R$なる$n$がある。これは$na - b > 0$を意味する。よって、$G$は$0$でないarchimedian順序群であり、階数$1$である。
\end{proof}
