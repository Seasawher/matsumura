\documentclass[12pt,dvipdfmx]{jsarticle}%文字サイズが12ptのjsarticle

%%%%%%%%%%%%%%%%%%%%%%%%%%%%%%%%%%%%%%%%%%%%%%%%%%%%%%%
%%  パッケージ                                        %%
%%%%%%%%%%%%%%%%%%%%%%%%%%%%%%%%%%%%%%%%%%%%%%%%%%%%%%%
%使用しないときはコメントアウトしてください
\usepackage{amsthm}%定理環境
\usepackage{framed}%文章を箱で囲う
\usepackage{amsmath,amssymb}%数式全般
\usepackage[dvipdfmx]{graphicx}%図の挿入
%\usepackage{tikz}%描画
\usepackage{titlesec}%見出しの見た目を編集できる
\usepackage[dvipdfmx, usenames]{color}%色をつける
%\usepackage{tikz-cd}%可換図式
%\usepackage{mathtools}%数式関連
\usepackage{amsfonts}%数式のフォント
\usepackage[all]{xy}%可換図式
\usepackage{mathrsfs}%花文字
\usepackage{comment}%コメント環境
\usepackage{picture}%お絵かき
\usepackage{url}%URLを出力
\usepackage[dvipdfmx]{hyperref}%ハイパーリンク
\usepackage{pxjahyper}%日本語しおりの文字化けを防ぐ


%%%%%%%%%%%%%%%%%%%%%%%%%%%%%%%%%%%%%%%%%%%%%%%%%%%%%%%
%%  表紙                                             %%
%%%%%%%%%%%%%%%%%%%%%%%%%%%%%%%%%%%%%%%%%%%%%%%%%%%%%%%
\makeatletter
\newcommand{\thickhrulefill}{\leavevmode \leaders \hrule height 1pt\hfill \kern \z@}
\renewcommand{\maketitle}{\begin{titlepage}%
    \let\footnotesize\small
    \let\footnoterule\relax
    \parindent \z@
    \reset@font
    \null\vfil
    \begin{flushleft}
      \huge \@title
    \end{flushleft}
    \par
    \hrule height 4pt
    \par
    \begin{flushright}
      \LARGE \@author \par
    \end{flushright}
    \vskip 60\p@
    \vfil\null
    \begin{flushright}
        {\small \@date}%
    \end{flushright}
  \end{titlepage}%
  \setcounter{footnote}{0}%
}
\makeatother

%%%%%%%%%%%%%%%%%%%%%%%%%%%%%%%%%%%%%%%%%%%%%%%%%%%%%%%
%%  sectionの装飾                                     %%
%%%%%%%%%%%%%%%%%%%%%%%%%%%%%%%%%%%%%%%%%%%%%%%%%%%%%%%
\titleformat{\section}[block]
{}{}{0pt}
{
  \colorbox{black}{\begin{picture}(0,10)\end{picture}}
  \hspace{0pt}
  \normalfont \Large\bfseries
  \hspace{-4pt}
}
[
\begin{picture}(100,0)
  \put(3,18){\color{black}\line(1,0){300}}
\end{picture}
\\
\vspace{-30pt}
]

%%%%%%%%%%%%%%%%%%%%%%%%%%%%%%%%%%%%%%%%%%%%%%%%%%%%%%%
%%  番号付き定理環境                                  %%
%%%%%%%%%%%%%%%%%%%%%%%%%%%%%%%%%%%%%%%%%%%%%%%%%%%%%%%
%注:defというコマンドはもうあります
\theoremstyle{definition}%定理環境のアルファベットを斜体にしない
\renewcommand{\proofname}{\textgt{証明}}%proof環境の修正

%%%%%%%%%%%%%%%%%%%%%%%%%%%%%%%%%%%%%%%%%%%%%%%%%%%%%%%
%%  番号なし定理環境                                  %%
%%%%%%%%%%%%%%%%%%%%%%%%%%%%%%%%%%%%%%%%%%%%%%%%%%%%%%%
%証明が付属するものは箱で囲う
\newtheorem*{lemma}{補題}
\newtheorem*{proposition}{命題}
\newtheorem*{claim}{主張}
\newcommand{\lem}[1]{\begin{oframed} \begin{lemma} #1 \end{lemma} \end{oframed}}%箱付きほだい
\newcommand{\prop}[1]{\begin{oframed} \begin{proposition} #1 \end{proposition} \end{oframed}}%箱付きめいだい

\newtheorem*{sol}{解答}
\newtheorem*{prob}{問題}
\newtheorem*{quo}{引用}
\newtheorem*{rem}{注意}


%%%%%%%%%%%%%%%%%%%%%%%%%%%%%%%%%%%%%%%%%%%%%%%%%%%%%%%
%%  左側に線を引く環境                                 %%
%%%%%%%%%%%%%%%%%%%%%%%%%%%%%%%%%%%%%%%%%%%%%%%%%%%%%%%
%leftbar環境の定義
\makeatletter
\renewenvironment{leftbar}{%
%  \def\FrameCommand{\vrule width 3pt \hspace{10pt}}%  デフォルトの線の太さは3pt
  \renewcommand\FrameCommand{\vrule width 1pt \hspace{10pt}}%
  \MakeFramed {\advance\hsize-\width \FrameRestore}}%
 {\endMakeFramed}
\newcommand{\barquo}[1]{\begin{leftbar} \begin{quo} #1 \end{quo} \end{leftbar}}%左線つき引用

%%%%%%%%%%%%%%%%%%%%%%%%%%%%%%%%%%%%%%%%%%%%%%%%%%%%%%%
%%  section                                        %%
%%%%%%%%%%%%%%%%%%%%%%%%%%%%%%%%%%%%%%%%%%%%%%%%%%%%%%%
\newcommand{\bfsubsection}[1]{\subsection*{\textbf{#1}}}%太字さぶせくしょん
\newcommand{\bfsection}[1]{\section*{\textbf{#1}}}

%%%%%%%%%%%%%%%%%%%%%%%%%%%%%%%%%%%%%%%%%%%%%%%%%%%%%%%
%%  色をつける                                      %%
%%%%%%%%%%%%%%%%%%%%%%%%%%%%%%%%%%%%%%%%%%%%%%%%%%%%%%%
\newcommand{\textblue}[1]{\textcolor{blue}{\textbf{#1}}}

%%%%%%%%%%%%%%%%%%%%%%%%%%%%%%%%%%%%%%%%%%%%%%%%%%%%%%%
%%  よく使う記号の略記                                 %%
%%%%%%%%%%%%%%%%%%%%%%%%%%%%%%%%%%%%%%%%%%%%%%%%%%%%%%%
%集合
\newcommand{\sm}{\setminus}%集合差
\newcommand{\setmid}[2]{\left\{ #1 \mathrel{} \middle| \mathrel{} #2 \right\}}%集合の内包記法
\newcommand{\tm}{\times}%直積

%解析
\newcommand{\abs}[1]{\left \lvert #1 \right \rvert}%絶対値
\newcommand{\norm}[1]{\left \lVert #1 \right \rVert}%ノルム
\newcommand{\I}{\sqrt{-1}}%虚数単位。\iは既にある。

%線形代数
\newcommand{\transpose}[1]{\, {\vphantom{#1}}^t\!{#1}}%行列の転置
\newcommand{\single}{\{ 0 \}}%0のシングルトン

%位相
\newcommand{\clsub}{\subset_{\text{closed}}}%閉部分集合
\newcommand{\opsub}{\subset_{\text{open}}}%開部分集合

\newcommand{\wt}[1]{\widetilde{#1}}%わいどちるだあ
\newcommand{\wh}[1]{\widehat{#1}}%わいどはっと
\newcommand{\To}{\Rightarrow}%ならば
\newcommand{\ol}[1]{\overline{#1}}%オーバーライン
\newcommand{\la}[1]{\overleftarrow{#1}}%上付き左矢印
\newcommand{\ra}[1]{\overrightarrow{#1}}%上付き右矢印
\newcommand{\xto}[1]{\xrightarrow{#1}}%上に文字がついてる右向き矢印

\newcommand{\st}{\; \; \text{s.t.} \; \;}%論理式でつかうsuch that
\newcommand{\f}[2]{\frac{#1}{#2}}%分数の略記
\newcommand{\ts}{\otimes}%テンソル積(tensor)




%%%%%%%%%%%%%%%%%%%%%%%%%%%%%%%%%%%%%%%%%%%%%%%%%%%%%%%
%%  演算子                                            %%
%%%%%%%%%%%%%%%%%%%%%%%%%%%%%%%%%%%%%%%%%%%%%%%%%%%%%%%
\DeclareMathOperator{\rank}{rk}%行列のランク
\DeclareMathOperator{\re}{Re}
\DeclareMathOperator{\gal}{Gal}%ガロア群
\DeclareMathOperator{\Hom}{Hom}%homはもうある
\DeclareMathOperator{\ind}{Ind}
\DeclareMathOperator{\tr}{Trace}
\DeclareMathOperator{\map}{Map}
\DeclareMathOperator{\Zar}{Zar}%ザリスキ・リーマン空間
\DeclareMathOperator{\Aut}{Aut}%自己同型群
\DeclareMathOperator{\height}{ht}%htはもうある
\renewcommand{\Im}{\operatorname{Im}}%写像の像。像対象。虚部が出力できなくなった。
\DeclareMathOperator{\Ker}{Ker}%写像の核。Abel圏の核対象。
\DeclareMathOperator{\im}{im}%イメージ、像。Abel圏では写像の方。
\DeclareMathOperator{\Coker}{Coker}%余核
\DeclareMathOperator{\trdeg}{tr\text{.}deg}%超越次数


%環
\DeclareMathOperator{\Frac}{Frac}%商体をとる操作
\DeclareMathOperator{\Spec}{Spec}%スペクトル
\DeclareMathOperator{\Supp}{Supp}%台
\DeclareMathOperator{\ann}{ann}%アナイアレーター
\DeclareMathOperator{\Ass}{Ass}%素因子
\DeclareMathOperator{\rad}{rad}%ヤコブソン根基の紛らわしい書き方。あるいは根基。

%圏
\DeclareMathOperator*{\llim}{\varprojlim}%極限。逆極限。射影極限。
\DeclareMathOperator*{\rlim}{\varinjlim}%余極限。順極限。入射極限。

%%%%%%%%%%%%%%%%%%%%%%%%%%%%%%%%%%%%%%%%%%%%%%%%%%%%%%%
%%  黒板太字(blackboard bold)                         %%
%%%%%%%%%%%%%%%%%%%%%%%%%%%%%%%%%%%%%%%%%%%%%%%%%%%%%%%
\newcommand{\bba}{{\mathbb A}}
\newcommand{\bbb}{{\mathbb B}}
\newcommand{\bbc}{{\mathbb C}}
\newcommand{\bbd}{{\mathbb D}}
\newcommand{\bbe}{{\mathbb E}}
\newcommand{\bbf}{{\mathbb F}}
\newcommand{\bbg}{{\mathbb G}}
\newcommand{\bbh}{{\mathbb H}}
\newcommand{\bbi}{{\mathbb I}}
\newcommand{\bbj}{{\mathbb J}}
\newcommand{\bbk}{{\mathbb K}}
\newcommand{\bbl}{{\mathbb L}}
\newcommand{\bbm}{{\mathbb M}}
\newcommand{\bbn}{{\mathbb N}}
\newcommand{\bbo}{{\mathbb O}}
\newcommand{\bbp}{{\mathbb P}}
\newcommand{\bbq}{{\mathbb Q}}
\newcommand{\bbr}{{\mathbb R}}
\newcommand{\bbs}{{\mathbb S}}
\newcommand{\bbt}{{\mathbb T}}
\newcommand{\bbu}{{\mathbb U}}
\newcommand{\bbv}{{\mathbb V}}
\newcommand{\bbw}{{\mathbb W}}
\newcommand{\bbx}{{\mathbb X}}
\newcommand{\bby}{{\mathbb Y}}
\newcommand{\bbz}{{\mathbb Z}}

%%%%%%%%%%%%%%%%%%%%%%%%%%%%%%%%%%%%%%%%%%%%%%%%%%%%%%%
%%  よく使う黒板太字                                  %%
%%%%%%%%%%%%%%%%%%%%%%%%%%%%%%%%%%%%%%%%%%%%%%%%%%%%%%%
\newcommand{\Z}{\bbz}
\newcommand{\A}{\bba}
\newcommand{\Q}{\bbq}
\newcommand{\R}{\bbr}
\newcommand{\C}{\bbc}
\newcommand{\F}{\bbf}
\newcommand{\N}{\bbn}
\renewcommand{\P}{\bbp}%パラグラフ記号が出力できなくなった

%%%%%%%%%%%%%%%%%%%%%%%%%%%%%%%%%%%%%%%%%%%%%%%%%%%%%%%
%%  カリグラフィー                                %%
%%%%%%%%%%%%%%%%%%%%%%%%%%%%%%%%%%%%%%%%%%%%%%%%%%%%%%%
\newcommand{\cala}{\mathcal{A}}
\newcommand{\calb}{\mathcal{B}}
\newcommand{\calc}{\mathcal{C}}
\newcommand{\cald}{\mathcal{D}}
\newcommand{\calf}{\mathcal{F}}
\newcommand{\calm}{\mathcal{M}}
\newcommand{\caln}{\mathcal{N}}
\newcommand{\calo}{\mathcal{O}}
\newcommand{\cals}{\mathcal{S}}

%%%%%%%%%%%%%%%%%%%%%%%%%%%%%%%%%%%%%%%%%%%%%%%%%%%%%%%
%%  ギリシャ文字(Greek letters)小文字                 %%
%%%%%%%%%%%%%%%%%%%%%%%%%%%%%%%%%%%%%%%%%%%%%%%%%%%%%%%
%コマンドが5字以上のもの
\newcommand{\gra}{{\alpha}}
\newcommand{\grg}{{\gamma}}
\newcommand{\grd}{{\delta}}
\newcommand{\gre}{{\epsilon}}
\newcommand{\grt}{{\theta}}
\newcommand{\grk}{{\kappa}}
\newcommand{\grl}{{\lambda}}
\newcommand{\grs}{{\sigma}}
\newcommand{\gru}{{\upsilon}}
\newcommand{\gro}{{\omega}}

\newcommand{\ve}{{\varepsilon}}
\newcommand{\vp}{{\varphi}}

%%%%%%%%%%%%%%%%%%%%%%%%%%%%%%%%%%%%%%%%%%%%%%%%%%%%%%%
%%  ギリシャ文字(Greek letters)大文字                 %%
%%%%%%%%%%%%%%%%%%%%%%%%%%%%%%%%%%%%%%%%%%%%%%%%%%%%%%%
%コマンドが5字以上のもの
\newcommand{\grG}{{\Gamma}}
\newcommand{\grD}{{\Delta}}
\newcommand{\grT}{{\Theta}}
\newcommand{\grL}{{\Lambda}}
\newcommand{\grS}{{\Sigma}}
\newcommand{\grU}{{\Upsilon}}
\newcommand{\grO}{{\Omega}}

%%%%%%%%%%%%%%%%%%%%%%%%%%%%%%%%%%%%%%%%%%%%%%%%%%%%%%%
%%  フラクトゥール                                  %%
%%%%%%%%%%%%%%%%%%%%%%%%%%%%%%%%%%%%%%%%%%%%%%%%%%%%%%%
\newcommand{\fraka}{\mathfrak{a}}
\newcommand{\frakm}{\mathfrak{m}}
\newcommand{\frakp}{\mathfrak{p}}
\newcommand{\frakq}{\mathfrak{q}}

%%%%%%%%%%%%%%%%%%%%%%%%%%%%%%%%%%%%%%%%%%%%%%%%%%%%%%%
%%  花文字                                          %%
%%%%%%%%%%%%%%%%%%%%%%%%%%%%%%%%%%%%%%%%%%%%%%%%%%%%%%%
%大文字しかどうせ使わない
\newcommand{\scra}{\mathscr{A}}
\newcommand{\scrf}{\mathscr{F}}
\newcommand{\scrg}{\mathscr{G}}
\newcommand{\scrs}{\mathscr{S}}

%%%%%%%%%%%%%%%%%%%%%%%%%%%%%%%%%%%%%%%%%%%%%%%%%%%%%%%
%%  太字                                            %%
%%%%%%%%%%%%%%%%%%%%%%%%%%%%%%%%%%%%%%%%%%%%%%%%%%%%%%%
\newcommand{\Mod}[1]{\textbf{Mod}\mathbf{#1}}%加群の圏


\begin{document}

\title{松村英之『可換環論』}
\author{\url{https://seasawher.hatenablog.com/} \\ @seasawher}
\date{\today}
\maketitle


\bfsection{\S 4. 局所化とスペクトル}

\bfsubsection{定理4.4 直前}
\barquo{
$P$に$P \cap A$を対応させる写像$\Spec (B) \to \Spec (A)$を${}^a f$と書く。容易にわかるように$({}^af)^{-1}(V(I)) = V(IB)$であるから、${}^af$は連続である。
}
\begin{proof}
  同値変形をおこなうと
  \begin{align*}
    P \in ({}^af)^{-1}(V(I)) &\iff ({}^af)(P) \in V(I) \\
    &\iff I \subset f^{-1}(P) \\
    &\iff f(I) \subset P \\
    &\iff IB \subset P \\
    &\iff P \in V(IB)
  \end{align*}
  が判る。
\end{proof}



\bfsubsection{定理 4.4 直前}
\barquo{
たとえば1点からなる集合$\{ \frakp \}$が$\Spec A$の閉集合であるためには、$\frakp$が極大イデアルであることが必要十分である。(一般に$\{ \frakp \}$の閉包$\ol{\{ \frakp \} }$は$V(\frakp)$に等しい。)
}
\begin{proof}
$\{ \frakp \}$を含む閉集合$V(I)$が与えられたとする。このとき$\frakp \in V(I)$であるから$I \subset \frakp$である。よって$V(\frakp) \subset V(I)$である。ゆえに$V(\frakp)$は$\{ \frakp \}$を含む閉集合$V(I)$の共通部分に含まれる閉集合なので$\ol{\{ \frakp \}} = V(\frakp)$である。

したがって$\{ \frakp \}$が閉集合であることは$\{\frakp\}=V(\frakp)$と同じことであるが、$\frakp$は真のイデアルだから$\frakp$を含む極大イデアルがあるはずで、よって$\frakp$は極大イデアルである。逆に$\frakp$が極大イデアルであれば閉点となることはあきらか。
\end{proof}




\bfsubsection{定理 4.4 直前}
\barquo{
$M$が有限生成なら、$M = A\gro_1 + \cdots + A\gro_n$とすると
\begin{align*}
  &\frakp \in \Supp (M) \iff M_{\frakp} \neq 0 \iff \exists i : M_{\frakp} \text{で$\gro_i \neq 0$} \iff \\
  &\exists i : \ann(\gro_i) \subset \frakp \iff \ann(M) = \bigcap_{i = 1}^n \ann(\gro_i) \subset \frakp
\end{align*}
であるから、$\Supp (M)$は$\Spec (A)$の閉集合$V(\ann (M))$に等しい。
}
\begin{rem}
  $M$が有限生成であるという仮定は
  \[
\bigcap_{i = 1}^n \ann(\gro_i) \subset \frakp \To \exists i : \ann(\gro_i) \subset \frakp
  \]
  を示すところで必要。$M$が有限生成でないときには、
  \[
  A = \Z \quad M = \bigoplus_{p : \text{prime}} \Z / p\Z
  \]
  が反例になる。素数$q$について
  \begin{align*}
    M_q &= \bigoplus_{p : \text{prime}} (\Z / p\Z \otimes_{\Z} \Z_q) \\
    &= \bigoplus_{p : \text{prime}} (\Z_q / p\Z \otimes_{\Z} \Z_q) \\
    &=  \Z_q / (q\Z \otimes_{\Z} \Z_q) \\
    &= (\Z/ q\Z)\text{の商体} \\
    &= \Z/ q\Z \\
    &\neq 0
  \end{align*}
  なので、$\Supp M = \Spec A \setminus \single$である。これは開でも閉でもない。
\end{rem}




\bfsubsection{定理 4.10}
\barquo{
$M$が有限表示加群ならば、$U_F = \setmid{\frakp \in \Spec (A)}{ \text{ $M_{\frakp}$は自由$A_{\frakp}$加群 }  }$は$\Spec (A)$の開集合である。
}
\begin{proof}
  $M_{\frakp}$が自由$A_{\frakp}$加群だとし、$\gro_1, \cdots , \gro_r$をその基底とする。(i)の証明の最後の注意により、$\frakp \in \Spec (A)$のある近傍$D(a)$が存在して、$D(a)$の各点$\frakq$において$\gro_1, \cdots , \gro_r$が$M_{\frakq}$を生成する。以降$B = A_a$、$N = M \otimes_A A_a$とする。

  このとき
  \[
  \forall \frakq \in D(a) \quad M_{\frakq} / \sum A_{\frakq} \gro_i = 0
  \]
  であるが、$B_{\frakq B}=A_{\frakq}$により
  \begin{align*}
    \forall \frakq \in D(a) \quad 0 &= (M / \sum A \gro_i) \otimes_A A_{\frakq} \\
    &=  (M / \sum A \gro_i) \otimes_A B_{\frakq B} \\
    &= (M / \sum A \gro_i) \otimes_A B \otimes_B B_{\frakq B} \\
    &= (N / \sum B \gro_i) \otimes_B B_{\frakq B}
  \end{align*}
  が成り立つ。$\Spec (B)= \setmid{\frakq B}{\frakq \in D(a)}$なので、定理4.6により
  \[
N / \sum B \gro_i = 0
  \]
  が結論できる。

  ここで$\vp \colon A^r \to M$を$\vp(a_1, \cdots , a_r) = \sum a_i \gro_i$で定めると
  \[
  \xymatrix{
  0 \ar[r] & \Ker (\vp \otimes B) \ar[r] & B^r \ar[r]^{\vp \otimes B}  & N \ar[r] & 0
  }
  \]
  は$B$加群の完全系列。$A$加群の完全系列だと思ってこれに$A_{\frakp}$をテンソルすると、$a$の定義により$B \otimes_A A_{\frakp} = A_{\frakp}$なので
  \[
  \xymatrix{
    0 \ar[r] & \Ker (\vp \otimes B)  \otimes_A A_{\frakp} \ar[r] & A_{\frakp}^r \ar[r]^{\vp \otimes A_{\frakp}}  & M_{\frakp} \ar[r] & 0
  }
  \]
  は$A_{\frakp}$加群の完全系列。よって$\vp \otimes A_{\frakp}$が同型であることにより
  \begin{align*}
    0 &= \Ker (\vp \otimes B) \otimes_A A_{\frakp} \\
    &= (\Ker \vp \otimes_A B) \otimes_A A_{\frakp} \\
    &= \Ker \vp \otimes_A A_{\frakp}
  \end{align*}
  であることが判る。

  $M$が有限表示であるという仮定から、$\Ker \vp$は有限生成なので(i)により
  \[
  \exists \frakp \in V \text{かつ} V \opsub \Spec A \text{かつ} \forall \frakq \in V \quad \Ker \vp \otimes_A A_{\frakq} = 0
  \]
  このとき任意の$\frakq \in V \cap D(a)$について$\vp \otimes A_{\frakq} \colon A_{\frakq}^r \xrightarrow{\vp \otimes A_{\frakq}} M_{\frakq} $は同型である。
\end{proof}


\newpage
\bfsection{\S 5. Hilbert零点定理と次元論初歩}


\bfsubsection{定理 5.6}
\barquo{
$r \geq \dim A$を示すには、$P$、$Q$が$k[X]=k[X_1, \cdots , X_n]$の素イデアルで$Q \supset P$、$Q \neq P$なら
\[
\trdeg_k k[X]/Q < \trdeg_k k[X]/P
\]
であることを示せば十分である。
}
\begin{rem}
先に$A = k[X]/P$としているが、この$P$は引用部の$P$とは別物である。
\end{rem}
\begin{proof}
  引用部を仮定する。$A$での素イデアルの鎖
  \[
  0=P_0 \subsetneq P_1 \subsetneq \cdots \subsetneq P_m
  \]
  を得る。よって引用部により
  \[
  \trdeg_k k[X]/\pi^{-1}(P_m) < \cdots < \trdeg_k k[X]/\pi^{-1}(P_1) < \trdeg_k A = r
  \]
  であるから$r \geq m$が判る。ゆえに、$\dim A \leq r$が結論される。
\end{proof}



\bfsubsection{定理 5.6}
\barquo{
$S = k[X_1, \cdots , X_r] - \single $とおくと$S$は乗法的集合で$P \cap S = \emptyset $、$Q \cap S = \emptyset$となる。
}
\begin{proof}
  $P \cap S \neq \emptyset$と仮定し、$f \in P \cap S$とする。このとき$\ol{f} \in k[X]/P$は$0$である。したがって$f(\gra_1, \cdots , \gra_r)=\ol{f(x_1, \cdots, x_r)} = 0$となる。ところが$\gra_1, \cdots , \gra_r$は$k$上代数的に独立だったから、これは矛盾。

  $Q \cap S = \emptyset$であることも同様。
\end{proof}



\bfsubsection{定理 5.6}
\barquo{
$P_{i}=Q_i \cap R$とおくと$P_i$は$S$と交わらない$R$の素イデアルで、したがって$\trdeg_k R/P_{r-1} > 0$、よって$P_{r-1}$は$R$の極大イデアルではない。
}
\begin{rem}
  $\trdeg_k R/P_{r-1} > 0$を示そう。$\ol{x}_1 \in R/P_{r-1}$が$k$上代数的でないことをいえばよい。$\ol{x}_1$が$k$上代数的であると仮定する。このとき、
  \[
  \sum_{i=0}^n a_i \ol{x}_1^i =0
  \]
  なる、$a_i \in k$がある。とくに$a_n \neq 0$となるようにとることができる。したがって
  \[
    \sum_{i=0}^n a_i x_1^i \in P_{r-1} \cap S = \emptyset
  \]
  となり矛盾である。
\end{rem}


\newpage
\bfsection{\S 6 素因子と準素分解}

\bfsubsection{\S 6 冒頭}
\barquo{
$A$の素イデアル$P$が$M$の適当な元$x$のannihilatorとなるとき、$P$を$M$の素因子という。
}
\begin{rem}
この定義は、$A/P \hookrightarrow M$が成り立つことと同値である。証明はやさしいので各自試みよ。
\end{rem}



\bfsubsection{定理 6.2}
\barquo{
$\Spec (A_S)$を$\Spec(A)$の部分集合とみなすとき、$\Ass_A(N)= \Ass_{A_S}(N)$である。$A$がネーター環なら、$A$加群$M$に対し$\Ass(M_S)=\Ass(M) \cap \Spec(A_S)$が成り立つ。
}
\begin{rem}
  本文を読んでも、何を主張していてどこでネーター性を使っているのか分かりにくい。次のようにいくつかの補題に分けて証明する。
\end{rem}

\lem{
$N \in \Mod{A_S}$に対して、次は可換。
\[
\xymatrix{
N \ar[r]^{id} \ar[d]_{\ann} & N \ar[d]^{\ann} \\
P(A_S) \ar[r]^b & P(A)
}
\]
すなわち、任意の$s \in N$に対して
\[
\ann_A(x) = \ann_{A_S}(x) \cap A
\]
が成り立つ。
}

\begin{proof}
$a \in A$とする。このとき
\begin{align*}
  a \in \ann_A{x} &\iff ax=0 \\
  &\iff \f{a}{1} \cdot x = 0 \\
  &\iff \f{a}{1} \in \ann_{A_S}(x) \\
  &\iff a \in \ann_{A_S}(s) \cap A
\end{align*}
が成り立つ。
\end{proof}

\lem{
  $\Spec(A \setminus S)$で$\setmid{\frakp \in \Spec(A)}{\frakp \cap S = \emptyset}$を表すことにする。このとき次は可換。
  \[
  \xymatrix{
  \Mod{A_S} \ar[r] \ar[dd]_{\Ass} & \Mod{A} \ar[d]^{\Ass} \\
  {} & P(\Spec (A)) \\
  P(\Spec (A_S)) \ar[r]^{b_*} & P(\Spec (A \setminus S)) \ar[u]_{i_*}
  }
  \]
  すなわち、任意の$N \in \Mod{A_S}$に対して
  \[
  \Ass_A(N) = \Ass_{A_S}(N)
  \]
  が成り立つ。
}

\begin{proof}
  任意の$\frakp \in \Spec(A)$について
  \begin{align*}
    \frakp \in \Ass_A(N) &\iff \exists x \in N \st \frakp = \ann_A(x) \\
    &\iff \frakp = \ann_A(x)  \; \text{かつ} \; \frakp \in \Spec(A \setminus S) \\
    &\iff \frakp = \ann_{A_S}(x) \cap A \; \text{かつ} \; \frakp \in \Spec(A \setminus S) \\
    &\iff \frakp A_S \in \Ass_{A_S}(N) \\
    &\iff \frakp \in \Ass_{A_S}(N)
  \end{align*}
  が成り立つことから判る。
\end{proof}

\lem{
    $A$加群$M$について次は可換。
    \[
    \xymatrix{
    M_S \ar[d]_{\ann} & M \ar[l] \ar[d]^{\ann} \\
    P(A_S) & P(A) \ar[l]^g
    }
    \]
    すなわち$x \in M$について
    \[
    \ann_A(x) A_S = \ann_{A_S}(x/1)
    \]
}

\begin{proof}
  $a/s \in A_S$に対して
  \begin{align*}
    \f{a}{s} \in \ann_A(x) A_S &\iff \exists t \in S \st ta \in \ann_A(x) \\
    &\iff tax = 0 \\
    &\iff ax = 0 \in M_S \\
    &\iff a \in \ann_{A_S}(x)
  \end{align*}
\end{proof}





\lem{
$A$がNoether環、$M$が$A$加群で、$x \in M$なら
\[
\ann_{A_S}(x/1) \cap A = \ann_A(tx)
\]
なる$t \in S$がある。
}

\begin{proof}
  $\ann_{A_S}(x/1) \cap A = \ann_A(x/1)$なので、$\ann_A(x/1) = \ann_A(tx)$なる$t \in S$があることを示せば十分である。

  $a \in \ann_A(x/1)$とする。このとき$a \cdot \f{x}{1} = 0 \; \in  M_S$であるから、ある$s \in S$が存在して$sax=0 \; \in M$が成り立つ。ゆえに$a \in \ann_A(sx)$である。$a$の選び方により$s$は変わるので$\ann_A(x/1) \subset \bigcup_{s \in S} \ann_A(sx)$である。逆は明らかだから、
  \[
  \ann_A(x/1) = \bigcup_{s \in S} \ann_A(sx)
  \]
  である。$A$はNoetherなので集合族$\setmid{\ann_A(sx)}{s \in S}$には極大元$\ann_A(tx)$が存在する。すると任意の$s \in S$について、$S$が積閉集合であることと極大性により$\ann_A(tx) = \ann_A(stx) \supset \ann_A(sx)$が判る。したがって$\ann_A(tx)$は実は最大元であり、
  \[
  \ann_A(x/1) = \ann_A(tx)
  \]
  が成り立つ。
\end{proof}


\lem{
    $A$がNoether環なら、次は可換。
    \[
    \xymatrix{
    \Mod{A_S} \ar[dd]_{\Ass} & \Mod{A} \ar[l] \ar[d]^{\Ass}  \\
    {} & P(\Spec (A)) \ar[d]^{i^*} \\
    P(\Spec (A_S)) \ar[r]^{b_*} & P(\Spec (A \setminus S))
    }
    \]
    すなわち、$A$加群$M$について
    \[
    \Ass_{A_S}(M_S) = \Ass_A(M) \cap \Spec(A_S)
    \]
    が成り立つ。
}

\begin{proof}
  $\frakp \in \Spec (A \setminus S)$とする。このとき
  \begin{align*}
    \frakp \in \Ass_A(M) &\iff \exists x \in M \st \frakp = \ann_A(x) \\
    &\iff \exists s \in S \st \frakp = \ann_A(s \cdot x) \\
    &\iff \frakp A_S = \ann_{A_S}(x/1) \\
    &\iff \frakp A_S \in \Ass_{A_S}(M_S)
  \end{align*}
  であるから、示したいことが言えた。
\end{proof}


\bfsubsection{定理 6.3}
\barquo{
$\Ass(M) \subset \Ass(M') \cup \Ass(M'')$
}
\begin{rem}
  等号が成り立たない例としては次の完全系列がある。
  \[
  \xymatrix{
  0 \ar[r] & \Z \ar[r]^{\times 12} & \Z \ar[r] & \Z / 12 \Z \ar[r] & 0
  }
  \]
  このとき$\Ass(\Z) = \{ (0) \} $であるが$\Ass( \Z / 12 \Z) = \{ (2), (3)\}$であるので等号が成立しない。
\end{rem}


\bfsubsection{定理 6.5}
\barquo{
$\Ass(A/P)=\{ P \}$であることに注意すればよい。
}
\begin{proof}
$\{ P \} \subset \Ass(A/P)$は、$P=\ann(1)$よりあきらか。逆に$Q \in \Ass(A/P)$であるとする。このとき素因子の定義から$Q = \ann(\ol{x})$なる$\ol{x} \in A/P$がある。したがって$P \subset Q$である。一方で$y \in Q$とすると$yx \in P$が従う。$x$は$P$の元ではあり得ないので、$P$が素イデアルであることより$y \in P$である。よって$P \supset Q$
、すなわち$P = Q$である。
\end{proof}





\bfsubsection{定理 6.5}
\barquo{
iii) $P$を$\Supp(M)$の極小元として$P \in \Ass(M)$を示せばよい。
}
\begin{rem}
  $\Ass(M)$における極小元が$\Supp(M)$でも極小元であることも確かめる必要がある。

  いま$P \in \Ass(M)$が$\Ass(M)$において極小であるとする。任意に$Q \subset P$なる$Q \in \Supp(M)$が与えられたとしよう。このとき$M_Q \neq 0$であり、$A$はNoetherなので$\Ass_{A_Q}(M_Q) \neq \emptyset$である。ゆえに
  \begin{align*}
  \emptyset \neq \Ass_{A_Q}(M_Q) &= \Ass_A(M) \cap \Spec (A_Q) \\
  &\subset \Ass_A(M) \cap \Spec (A_P) \\
  &= \{ P \}
\end{align*}
が成り立つので$P \in \Ass_{A_Q}(M_Q) \subset \Spec (A_Q)$でなくてはならない。よって$P = Q$である。$Q$は任意だったので、$P$は$\Supp(M)$でも極小ということが示せた。
\end{rem}




\bfsubsection{定理 6.5 直後}
\barquo{
$A$をNoether環、$M$を有限$A$加群とする。$\Supp(M)$の極小元を$P_1, \cdots , P_r$とすれば$\Supp(M)= V(P_1) \cup \cdots \cup V(P_r)$であり
}
\begin{proof}
  $A$がNoether環で$M$が有限$A$加群という仮定から、$\Supp(M)$の極小元は有限個しかないので、$\Supp(M)$の極小元を$P_1, \cdots , P_r$とおくことができる。このとき$\Supp(M) \subset V(P_1) \cup \cdots \cup V(P_r)$である。逆に、$Q \in V(P_1) \cup \cdots \cup V(P_r)$なら$Q \supset P_i$なる$i$がある。すると
  \[
  0 \neq M_{P_i} = M_Q \otimes_A A_{P_i}
  \]
  より$M_Q \neq 0$なので$Q \in \Supp(M)$である。
\end{proof}




\bfsubsection{定理 6.6}
\barquo{
前定理により$\Supp (M/N) = V(\frakp)$、したがって$P = \sqrt{\ann(M/N)}$である。
}
\begin{proof}
  $M$は有限生成なので、$M/N$も有限生成。よって
  \[
  \Supp (M/N) = V(\ann(M/N))
  \]
  が成り立つ。したがって
  \begin{align*}
    \sqrt{\ann(M/N)} &= \bigcap_{\frakq \in V(\ann(M/N))} \frakq \\
    &= \bigcap_{\frakq \in V(\frakp)} \frakq \\
    &= \sqrt{\frakp} \\
    &= \frakp
  \end{align*}
 が判る。
\end{proof}





\bfsubsection{定理 6.8 (i)}
\barquo{
$x \in K_i$, $x \neq 0$なら$\ann(x)=P_i$であるから
}
\begin{rem}
  この部分は定理6.3の証明と同様。
\end{rem}



\bfsubsection{定理 6.8 (ii)}
\barquo{
一方$N : M$は$P_1$に属する準素イデアルだから
}
\begin{rem}
  要するに$\sqrt{\ann(M/N)} = P_1$ということである。
\end{rem}



\bfsubsection{定理 6.8 (iii)}
\barquo{
$i > 1$に対し$\ann(M/N_i)$は$P_i$のべきを含み、$P_i \not\subset P_1$であるから$(M/N_i)_P=0$であり、
}
\begin{proof}
  $P_i$は$\ann(M/N_i)$の根基で、$A$はNoetherなので$P_i^{v} \subset \ann(M/N_i)$なる$v \geq 1$がある。$P = P_1$は極小素因子としてとってきたので$P_i \not\subset P_1$であり、$P$は素イデアルなので$P_i^v \cap P^c \neq \emptyset$である。よって$\ann(M/N_i) \cap P^c \neq \emptyset$である。したがって$(M/N_i)_P=0$がわかる。
\end{proof}




\bfsubsection{定理 6.8 (iii)}
\barquo{
したがって$\vp_P^{-1}(N_P) = \vp_P^{-1}((N_1)_P)$であり、この右辺は容易に確かめられるように$N_1$に等しい。
}
\begin{proof}
  $\vp_P^{-1}((N_1)_P) = N_1$を示そう。$\vp_P^{-1}((N_1)_P) \supset N_1$はあきらかなので逆を示す。

  $x \in \vp_P^{-1}((N_1)_P)$とする。このとき$x/1 \in (N_1)_P$であるから、ある$s \in A \setminus P$が存在して$s x \in N_1$である。このときもし$x \not\in N_1$ならば、$s$は$M/N_1$-非正則ということになり、$s \in P$でなければならない。これは矛盾。
\end{proof}


\newpage
\bfsection{\S 7 平坦性}

\bfsubsection{\S 7 冒頭}
\barquo{
$\calf$が完全列なら$0 \to N_1 \to N_2 \to N_3 \to 0$のような形の列(いわゆる短完全列)に分解できるから、平坦性の定義において$\calf$として短完全列のみを考えればよい。
}
\begin{rem}
  まず次の補題を示す。
\end{rem}

\lem{
$M$は(短完全列について)平坦な$A$加群であるとし、$A$加群の準同型$f \colon N_1 \to N_2$があるとする。このとき次が成り立つ。
\begin{description}
  \item[(1)] $\Im f \otimes M = \Im(f \otimes M)$.
  \item[(2)] $\Ker f \otimes M = \Ker (f \otimes M)$.
\end{description}
}
\begin{rem}
  一般に加群の圏からAbel群への共変関手$F$が左完全ならそれは$\Ker$を保ち、右完全ならそれは$\Coker$を保つ。証明はやさしいが、[Rotman]\cite{Rotman} 命題5.25を参照のこと。
\end{rem}
\begin{proof}
  次の完全な行をもつ可換図式を考える。
  \[
  \xymatrix{
  {} & N_2 & {} \\
  N_1 \ar[r]^{\wt{f}} \ar[ur]^f & \Im f \ar[u] \ar[r] & 0
  }
  \]
  は完全。したがって$M$をテンソルすると
  \[
  \xymatrix{
  {} & N_2 \ts M & {} \\
  N_1 \ts M \ar[r]^{\wt{f} \ts M} \ar[ur]^{f \ts M} & \Im f \ts M  \ar[u] \ar[r] & 0
  }
  \]
  を得る。テンソル積の右完全性により
  \begin{align*}
    \Im f \ts M &= \Im (\wt{f} \ts M) \\
    &\cong \Im (f \ts M)
  \end{align*}
  が成り立つ。

  また、完全系列
  \[
  0 \to \Ker f \to N_1 \xto{f} N_2
  \]
  を考える。$M$をテンソルして、完全系列
  \[
  0 \to \Ker f \ts M \to N_1 \ts M \xto{f \ts M} N_2 \ts M
  \]
  を得る。したがって
  \begin{align*}
    \Ker (f \ts M) &= \Im (\Ker f \ts M \to N_1 \ts M) \\
    &\cong \Ker f \ts M
  \end{align*}
  である。
\end{proof}

\begin{proof}
  引用部の証明に戻る。短完全列について平坦な$A$加群$M$があり、完全系列
  \[
  \calf \colon  \cdots \to N_n \xrightarrow{f_n} N_{n+1} \to \cdots
  \]
  が与えられたとする。このとき
  \[
  \Ker( f_n \otimes M) = \Ker f_n \otimes M = \Im f_{n-1} \otimes M = \Im (f_{n-1} \otimes M)
  \]
  より$M$が平坦であることがわかる。
\end{proof}





\bfsubsection{定理 7.1}
\barquo{
一般に、$B$が$A$代数で$M$, $N$が$B$加群なら、テンソル積の構成法からわかるように、$M \otimes_B N$は$M \otimes_A N$を$\setmid{bx \otimes y - x \ts by}{x \in M, y \in N , b \in B}$で生成された部分加群で割った剰余加群である。
}
\begin{proof}
  $J \subset M \ts_A N$を$\setmid{bx \otimes y - x \ts by}{x \in M, y \in N , b \in B}$で生成された部分加群とする。このとき$M \ts_A N / J$に$B$加群の構造$B \times (M \ts_A N / J) \to M \ts_A N / J$を
  \[
  b \cdot (x \ts y) = bx \ts y = x \ts by
  \]
  により定めることができる。この構造が$M \ts_B N$の構造と一致することは直感的にはいかにもありそうなことだが、実際に普遍性を使って確認することができる。詳しいことは読者に任せる。
\end{proof}



\bfsubsection{定理 7.1}
\barquo{
最初の注意と仮定により$K_{P}=0$を得る。
}
\begin{proof}
  $0 \to N' \xto{f} N $は$\Mod{A}$において完全であるとする。このとき$K = \Ker (f \ts M)$とおくと
  \[
  0 \to K \to N' \ts_A M \xto{f \ts M} N \ts_A M
  \]
  は$\Mod{B}$において完全である。よって$B_P \in \Mod{B}$の平坦性により
  \[
    0 \to K_P \to N' \ts_A M_P \xto{f \ts M_P} N \ts_A M_P
  \]
  は$\Mod{B_P}$において完全。ゆえに
  \[
  K_P = \Ker (f \ts_A M_P) \cong \Ker (f \ts_A A_{\frakp} \ts_{A_{\frakp}}  M_P )
  \]
  であるが、$A_{\frakp} \in \Mod{A}$と$M_P \in \Mod{A_{\frakp}}$の平坦性により$K_P = 0$である。

  以上の議論では、最初の注意は使わなかった。
\end{proof}




\bfsubsection{定理 7.2}
\barquo{
$M \neq \frakm M \supset \ann(x) \cdot M$であるから$Ax \ts M \neq 0$.
}
\begin{proof}
  テンソル積の右完全性により
  \begin{align*}
      Ax \ts M &\cong A / \ann(x) \ts M \\
      &\cong M / \Im ( \ann(x) \ts M \to M)  \\
      &= M / \ann(x) \cdot M \\
      &\neq 0
  \end{align*}
  であることからわかる。
\end{proof}



\bfsubsection{定理 7.2 直後}
\barquo{
よって${}^ag \colon \Spec(C) \to \Spec(B)$の像は
\[
\setmid{P \in \Spec(B)}{P \supset \frakp B, P \cap f(S) = \emptyset } = \setmid{P \in \Spec(B)}{P \cap A = \frakp}
\]
すなわち${}^af^{-1}(\frakp)$であり、
}
\begin{proof}
  $g$の定義により次は可換。
  \[
  \xymatrix{
  C \ar[r]^I & B_S / \frakp B_S \\
  B \ar[u]^g \ar[r]^{\psi} & B_S \ar[u]_{\pi}
  }
  \]
  したがって$ I \circ g = \pi \circ \psi$なので${}^ag \circ {}^aI = {}^a\psi  \circ {}^a\pi$なので
  \begin{align*}
    \Im ({}^ag) &= \Im({}^a\psi  \circ {}^a\pi) \\
    &= \setmid{\psi^{-1}(P') }{P' \in \Spec(B_S), \frakp B_S \subset P' } \\
    &= \setmid{P \in \Spec(B)}{P \cap f(S) = \emptyset , \frakp B \subset P} &(\frakp B_S \subset P' \subsetneq B_S \text{による})
  \end{align*}
  である。ここで$P \cap f(S) = \emptyset \iff f^{-1}(P) \subset \frakp$であることと、$\frakp B \subset P \iff f^{-1}(P) \supset \frakp$であることに注意すると$\Im({}^ag) = \setmid{P \in \Spec(B)}{P \cap A = \frakp}$がわかる。
\end{proof}



\bfsubsection{定理 7.2 直後}
\barquo{
${}^ag$は$\Spec(C)$から${}^af^{-1}(\frakp)$の上への位相同形をひきおこす。
}
\begin{proof}
  次の図式は可換。
  \[
  \xymatrix{
  \Spec(C) \ar[d]_{{}^ag} & \Spec(B_S / \frakp B_S) \ar[l]_{{}^aI} \ar[d]^{{}^a\pi} \\
  {}^af^{-1}(\frakp) & V(\frakp B_S) \ar[l]_{{}^a\psi|_{V(\frakp B_S)}}
  }
  \]
  すると、${}^ag$以外すべて同相なので${}^ag$も同相。
\end{proof}


\bfsubsection{定理 7.3 (ii)}
\barquo{
なおさら$M_P / \frakm M_P = (M/ \frakm M)_P \neq 0$、
}
\begin{rem}
  $0 \neq M_P / \frakm M_P$が無断で使われているが、これは全射$M_P / \frakm M_P \to M_P / P M_P$の存在からいえる。
\end{rem}



\bfsubsection{定理 7.4 (i)}
\barquo{
これはi)の主張を意味する。
}
\begin{rem}
  $(N_1 \cap N_2) \ts M \subset (N_1 \ts M) \cap (N_2 \ts M)$はあきらか。また、$(N_1 \ts M) \cap (N_2 \ts M)$の元であれば$N \ts M \to (N \ts M)/(N_1 \ts M) \oplus (N \ts M)/(N_2 \ts M)$の核に入るので、逆の包含が言える。
\end{rem}




\bfsubsection{定理 7.6}
\barquo{
完全列$K \xto{i} A^n \xto{\vp} A^r$に$\ts M$を施して完全列
\[
K \ts M \xto{i \ts 1} M^n \xto{\vp_M} M^r
\]
が得られる。
}
\begin{proof}
次の完全列がある。
\begin{gather*}
  0 \to K \xto{i} A^n \xto{\wt{\vp} } \Im \vp \to 0 \\
  0 \to \Im \vp \xto{j} A^r \to \Coker \vp \to 0
\end{gather*}
これに$M$をテンソルすると、平坦性により次は完全。
\begin{gather*}
  0 \to K \ts M \xto{i \ts M} A^n \ts M \xto{\wt{\vp} \ts M} \Im \vp \ts M \to 0 \\
  0 \to \Im \vp \ts M \xto{j \ts M} M^r \to \Coker \vp \ts M \to 0
\end{gather*}
したがって
\begin{align*}
  \Ker \vp_M &= \Ker ((j \circ \wt{\vp}) \ts M ) \\
  &= \Ker ((j \ts M )(\wt{\vp} \ts M))  \\
  &= \Ker (\wt{\vp} \ts M)  \\
  &= \Im (i \ts M )
\end{align*}
がわかる。よって求める完全性がいえた。
\end{proof}



\bfsubsection{定理 7.7}
\barquo{
ここで$I = \setmid{a \in A}{a \gro \in N'}$とおけば
\[
0 \to N' \to N \to A/I \to 0
\]
という完全列が得られる。
}
\begin{rem}
  $\gro$倍が定める写像$A \to N / N'$の核は$I$なので$N / N' \cong A / I$である。
\end{rem}




\bfsubsection{定理 7.7 直後}
\barquo{
上の定理から定理6の逆が証明できる。
}
\begin{rem}
  つまり、環$A$と$A$加群$M$について
  \[
  M\text{が平坦} \iff \forall n,r \; \forall \vp \colon A^n \to A^r (\text{linear}) \quad \Ker \vp \ts M \to M^n \xto{\vp \ts M} M^r \text{が完全}
   \]
   が成り立つ。
\end{rem}




\bfsubsection{定理 7.10}
\barquo{
したがって$M$が有限生成または$\frakm$がべき零なら、$M$の任意の極小底(\S 2 参照)は$M$の基底になり、$M$は自由加群である。
}
\begin{proof}
$x_1, \cdots , x_n$を$M$の極小底とする。このとき像$\ol{x_1}, \cdots, \ol{x_n}$は$M/\frakm M$を生成する。もしこれが$A/\frakm$上1次独立でなければ、ある真部分集合$S \subsetneq \{ 1, \cdots , n \}$が存在して$\{ x_i \}_{i \in S}$は$M/\frakm M$の基底になる。
このとき$M = \sum_{i \in S} Ax_i + \frakm M$が成り立つ。

$M$が有限生成なら、このときNAKが適用できて$M = \sum_{i \in S} Ax_i$である。これは始めに極小底をとってきたことに反する。

$\frakm$がべき零なら、$N =  \sum_{i \in S} A x_i$とすると
\begin{align*}
  M &= N + \frakm M \\
  &= N + \frakm ( N + \frakm M) \\
  &= N + \frakm^2 M
\end{align*}
以下帰納的に、任意の$r$について$M = N + \frakm^r M$が成り立つので、巾零性により$M = N$であることがわかる。これは極小底をとってきたことに矛盾する。
\end{proof}


\bfsubsection{定理 7.11}
\barquo{
関手の準同形$\grl \colon F \to G$が$\grl (f \ts b) = b \cdot (f \ts 1_B) \; (f \in \Hom_A(M,N) , b \in B)$で定義できる。
}
\begin{proof}
  $\grl_M \colon F(M) \to G(M)$が自然変換であることを示せばよい。$g \colon M \to M'$が与えられたとする。
  \[
  \xymatrix{
  F(M) \ar[r]^{\grl_M} & G(M) \\
  F(M') \ar[u]^{g^* \ts 1_B} \ar[r]^{\grl_{M'}} & G(M') \ar[u]_{(g \ts 1_B)^*}
  }
  \]
  の可換性を言えばよい。

  $f' \in \Hom_A(M',N)$と$b \in B$に対して
  \begin{align*}
    \grl_M (g^* \ts 1_B) (f' \ts b) &= \grl_M (f' g \ts b) \\
    &= b \cdot (f'g \ts 1_B) \\
    (g \ts 1_B)^* \grl_{M'} (f' \ts b) &= (g \ts 1_B)^* (b \cdot (f' \ts 1_B)) \\
    &= b \cdot ((f' \ts 1_B) \circ (g \ts 1_B) ) \\
    &= b \cdot (f'g \ts 1_B)
  \end{align*}
  が成り立つ。よって示したいことがいえた。
\end{proof}




\bfsubsection{定理 7.11}
\barquo{
さて$F(A^p) = N^p \ts B$, $G(A^p) = (N \ts B)^p$だから
}
\begin{proof}
  $\Hom$関手については次の性質が成り立つことが知られている。
\begin{gather*}
  \Hom(A, \prod B_i) \cong \prod \Hom(A,B_i) \\
  \Hom (\bigoplus A_i , B) \cong \prod \Hom(A_i, B)
\end{gather*}
したがって、加群の圏において有限直和と有限直積は同型であるということから、求める式が得られる。
\end{proof}


\bfsubsection{定理 7.11}
\barquo{
したがって容易にわかるように左の$\grl$も同型となる。
}
\begin{rem}
  5-lemmaを適用すればよい。5-lemmaの主張は次の通り。
\end{rem}
\lem{
各行が完全であるような可換図式
\[
\xymatrix{
A_1 \ar[r] \ar[d]^{\vp_1} & A_2 \ar[r] \ar[d]^{\vp_2} & A_3 \ar[r]  \ar[d]^{\vp_3} &  A_4 \ar[r] \ar[d]^{\vp_4} & A_5 \ar[d]^{\vp_5} \\
B_1 \ar[r] & B_2 \ar[r] & B_3 \ar[r] & B_4 \ar[r] & B_5
}
\]
があり、次が成り立つとする。
\begin{description}
  \item[(1)] $\vp_4$と$\vp_2$は同型
  \item[(2)] $\vp_1$は全射
  \item[(3)] $\vp_5$は単射
\end{description}
このとき、$\vp_3$は同型である。
}
\begin{proof}
  ${}$[Osborne]\cite{Osborne}命題2.5を参照のこと。
\end{proof}


\newpage
\bfsection{\S 8 完備化とArtin-Reesの補題}

\bfsubsection{\S 8 冒頭}
\barquo{
このとき、$\calf$を$0$の近傍系として$M$は(加法に関し)位相群になる。
}

\begin{rem}
  次の補題による。
\end{rem}

\lem{
$G$を群、$\caln$を$G$の正規部分群よりなる空でない集合とする。もし$\caln$が性質
\[
H_1,H_2 \in \caln \; \text{なら} \; H_1 \cap H_2 \in \caln
\]
を満たせば、$B$を$gH \; (g \in G, H \in \caln)$という形の部分集合全体の集合とするとき、$B$は$G$の開基となる。また、$B$により定まる位相により$G$は位相群となる。
}
\begin{proof}
  ${}$[雪江]\cite{雪江3} 命題1.3.10を参照のこと。
\end{proof}

\bfsubsection{\S 8 冒頭}
\barquo{
$M$では和、差が連続であるのみならず、$A$の元$a$によるスカラー倍$x \mapsto ax$も$M$から$M$への連続写像になる。
}
\begin{proof}
  $a(x + M_{\grl}) \subset ax + M_{\grl}$による。
\end{proof}

\bfsubsection{\S 8 冒頭}
\barquo{
この位相が分離的(=Hausdorff)であるための必要十分条件は$\bigcap_{\grl} M_{\grl} = 0$である。
}
\begin{rem}
次の命題による。
\end{rem}

\prop{
($T_1$空間の特徴付け) \\
$X$を位相空間とする。このとき次は同値。
\begin{description}
  \item[(1)] $X$は$T_1$空間である。つまり任意の$x,y \in X$に対して、$x$のある近傍であって$y$を含まないものが存在する。
  \item[(2)] 任意の$x \in X$について、$\{x \} \clsub X$が成り立つ。
  \item[(3)] 任意の$x \in X$について、$x$のすべての近傍の共通部分が$\{x \}$に等しい。
\end{description}
}
\begin{proof}
  よく知られた事実であるため省略。
\end{proof}



\prop{
台集合が必ずしもHausdorffとは限らない位相群$G$について、次の主張は同値。
\begin{description}
  \item[(1)] $G$は$T_1$空間。
  \item[(2)] $G$はHausdorff空間。
  \end{description}
}
\begin{proof}
  ${}$[FANF]\cite{FANF} 命題1.3を参照のこと。
\end{proof}

\prop{
$X$を位相空間とする。このとき次は同値。
\begin{description}
\item[(1)] $X$はHausdorff空間、つまり任意の$x , y \in X$に対して$x,y$の開近傍$U,V$であって$U \cap V = \emptyset$なるものが存在する。
\item[(2)] 任意の$x \in X$に対して、$x$のすべての閉近傍の共通部分が$\{x \}$に等しい。
\end{description}
}
\begin{proof}
    ${}$[内田]\cite{内田} 定理21.1を参照のこと。なお、この命題は引用部の証明には不要。単なるオマケである。
\end{proof}


\bfsubsection{\S 8 冒頭}
\barquo{
剰余加群$M/M_{\grl}$は(商空間の位相で)ディスクリート空間になる。
}
\begin{rem}
  $M \to M/M_{\grl}$による$0$の引き戻しは$M_{\grl}$で、これが開集合であることから。
\end{rem}


\bfsubsection{\S 8 冒頭}
\barquo{
自然な$A$線形写像$M \to \wh{M}$を$\psi$とおくと、$\psi$は連続であり、$\psi(M)$は$\wh{M}$で稠密である。
}
\begin{rem}
  積位相の定義から従う。
\end{rem}


\bfsubsection{\S 8 冒頭}
\barquo{
$p_{\grl} \colon M_{\grl} \to M/M_{\grl}$を射影とし$\Ker (p_{\grl}) = M_{\grl}^*$とおくと、$\wh{M}$の位相は$\calf^* = \{ M_{\grl}^*\}$で定義された線形位相と一致することが容易にわかる。
}
\begin{proof}
  $\grl < \mu$のとき次の図式は可換である。
  \[
  \xymatrix{
  {} & \wh{M} \ar[ld]_-{p_{\grl}} \ar[dr]^-{p_{\mu}} & {} \\
M/M_{\grl} & {} & M/M_{\mu} \ar[ll]_-{\vp_{\grl \mu}}
  }
  \]
  よって$\Ker (p_{\grl}) \supset \Ker(p_{\mu})$である。したがって、
  \[
  \Ker (p_{\mu}) = (\prod_{\grl \leq \mu} \{0\} \tm \prod_{\text{otherwise}} M/M_{\grl}) \cap \wh{M}
  \]
  が成り立つ。ゆえに、任意の$\mu \in \grL$に対して$0 \in U \subset \Ker(p_{\mu})$なる開集合$U$があること、そして任意の$0$の開近傍$U$に対して$0 \in \Ker(p_{\mu}) \subset U$なる$\mu \in \grL$があることが従う。
\end{proof}



\bfsubsection{\S 8 冒頭}
\barquo{
一般に、$\psi \colon M \to \wh{M}$が同形写像になるとき$M$を完備であるという。
}
\begin{rem}
  このとき
  \[
  \Ker \psi = \bigcap_{\grl } M_{\grl} = 0
  \]
  なので特に$M$は線形位相でHausdorff空間になる。
\end{rem}



\bfsubsection{\S 8 冒頭}
\barquo{
このとき$\llim (M/M_{\grl}) \cong \llim (M/M'_{\grg})$であることが容易にわかり、またこの同形は位相同形でもある。
}
\begin{proof}
  式を見やすくするため$\wh{M} = \llim (M/M_{\grl})$、$\wt{M} = \llim(M/M'_{\grg})$とおく。$\grl \in \grL$に対して$\pi_{\grl} \colon \wt{M} \to M/M_{\grl}$を、次の図式が可換になるように定める。
\[
\xymatrix{
\wt{M} \ar[r]^-{p_{\grd}} \ar[rd]_{\pi_{\grl}} & M/M'_{\grd} \ar[d]^{i_{\grd \grl}}  & \text{($\grd \in \grG, M'_{\grd} \subset M_{\grl}$ )} \\
{} & M/M_{\grl} & {}
}
\]
  すなわち$M'_{\grd} \subset M_{\grl}$なるある$\grd \in \grG$に対して
  $\pi_{\grl} = i_{\grd \grl} \circ p_{\grd}$により定める。これは、$\grd$の取り方によらずwell-definedである。($M'_{\grd'} \subset M_{\grl}$なる別の$\grd'$が与えられたとき、$\grG$は有向集合なので$\grd''> \grd, \grd'$なる$\grd''$がとれる。そうすると$ i_{\grd \grl} \circ p_{\grd}$も$i_{\grd' \grl} \circ p_{\grd'}$
  も$\grd''$で書けて、等しいことがわかる。)

  以下の図式を見ればわかるように、写像の族$\pi_{\grl}$は$\vp_{\grl \mu}$と可換である。
  \[
  \xymatrix{
  M/M'_{\grd} \ar[d]_{i_{\grd \mu}} & \wt{M} \ar[dl]_{\pi_{\mu}} \ar[l]_{p_{\grd}} \ar[r]^{p_{\grd}} \ar[rd]^{\pi_{\grl}} & M/M'_{\grd} \ar[d]^-{i_{\grd \grl}} \\
  M/M_{\mu} \ar[rr]^-{\vp_{\grl \mu}} & {} & M/M_{\grl}
  }
  \]
  \[
  (M'_{\grd} \subset M_{\mu} \subset M_{\grl})
  \]
したがって、次の図式
\[
\xymatrix{
\wt{M} \ar[d]_-{\pi_{\grl}} \ar[r]^I & \wh{M} \ar[dl]^-{p_{\grl}} \\
M/M_{\grl} & {}
}
\]
が可換になる$I \colon \wt{M} \to \wh{M}$がある。同様にして、次の図式
\[
\xymatrix{
\wh{M} \ar[d]_-{p_{\grl}} \ar[rd]^-{\pi_{\grd}} & {} & \wh{M} \ar[d]_-{\pi_{\grd}} \ar[r]^J & \wt{M} \ar[dl]^-{p_{\grd}} \\
M/M_{\grl} \ar[r]_-{j_{\grl \grd}} & M/M'_{\grd} & M/M'_{\grd} & {}
}
\]
\[
(M_{\grl} \subset M'_{\grd})
\]
が可換になるような$\pi_{\grd}$と$J$が存在する。このとき次は可換である。
\[
\xymatrix{
\wt{M} \ar[d]_-{p_{\grd}} \ar[r]^I & \wh{M} \ar[dl]^-{\pi_{\grd}} \\
M/M'_{\grd} & {}
}
\]
なぜならば、
\begin{align*}
  \pi_{\grd} \circ I &= j_{\grl \grd} \circ p_{\grl} \circ I &(M_{\grl} \subset M'_{\grd}) \\
  &= j_{\grl \grd} \circ \pi_{\grl} \\
  &= j_{\grl \grd} \circ i_{\grg \grl} \circ p_{\grg} &(M'_{\grg} \subset M_{\grl}) \\
  &= \vp_{\grd \grg} \circ p_{\grg} \\
  &= p_{\grd}
\end{align*}
であるからである。ゆえに、
\begin{align*}
  p_{\grd}JI &= \pi_{\grd} I \\
  &= p_{\grd}
\end{align*}
であるから、逆極限の普遍性により$JI = id$がいえる。逆も同様であり、$I,J$は連続だから位相同形であることも従う。

\end{proof}





\bfsubsection{定理 8.1 直前}
\barquo{
$\xi = (\xi_{\grl})_{\grl \in \grL} \in \wh{M}$が$\wh{N}$に属するための条件は各$\xi_{\grl}$が$N$の元で代表されること、いいかえればすべての$\grl$に対して$\xi \in \psi(N) + M^*_{\grl}$であることである。
}
\begin{proof}
  ($\To$) $\xi \in \wh{N}$と仮定する。このとき$\xi_{\grl} \in (M_{\grl} + N)/M_{\grl} = N/(N \cap M_{\grl})$の$N$への持ち上げ$\wt{\xi_{\grl}}$をとると、
  \[
  p_{\grl}(\xi - \psi(\wt{\xi_{\grl}})) = 0
  \]
  だから、$\xi \in \bigcap_{\grl} \psi(N) + M^*_{\grl}$がいえる。

  ($\Leftarrow$) $\xi \in \wh{M}$が任意の$\grl$に対して$\xi \in \psi(N) + M^*_{\grl}$を満たすとする。このとき、ある$x_{\grl} \in N$が存在して$\xi - \psi(x_{\grl}) \in M^*_{\grl}$である。つまり$\xi_{\grl} - \ol{x_{\grl}} = 0 \; \text{in} \; M/M_{\grl}$
  である。ゆえに$\xi_{\grl} \in (N+M_{\grl})/M_{\grl}$が成り立つので、$\xi \in \wh{N}$である。
\end{proof}




\bfsubsection{定理 8.1 直後}
\barquo{
$\vp_{\grg} \colon \wh{M} \to N/N_{\grg}$を$\wh{M} \to \wh{M} /M^*_{\grl} \to N/N_{\grg}$ (第1の矢は自然な射、第2の矢は$f$からひき起こされたもの) の合成で定義すると
}
\begin{rem}
  次の図式が可換になるように定める。
  \[
  \xymatrix{
  \wh{M} \ar[r]^-{p_{\grl}} \ar[d]_-{\vp_{\grg}} & M/M_{\grl} \ar[dl]^-{f_{\grl \grg}}     &  M \ar[r]^-f \ar[d]_{\pi_{\grl}} & N \ar[d]^{\pi_{\grg}} \\
  N/N_{\grg} & {}      &  M/M_{\grl} \ar[r]^-{f_{\grl \grg}} & N/N_{\grg}
  }
  \]
\end{rem}



\bfsubsection{定理 8.1 直後}
\barquo{
自然な射$N/N_{\grg'} \to N/N_{\grg}$を$\psi_{\grg \grg'}$で表せば、$\vp_{\grg} = \psi_{\grg \grg'} \circ \vp_{\grg'}$が成り立つことも見やすい。
}
\begin{rem}
  $M_{\grl } \subset f^{-1}(N_{\grg'}) \subset f^{-1}(N_{\grg})$なる$\grl$をとると
  \begin{align*}
    \psi_{\grg \grg'} \circ \vp_{\grg'} &= \psi_{\grg \grg'} \circ f_{\grg \grg'} \circ p_{\grl} \\
    &= f_{\grl \grg} \circ p_{\grl} \\
    &= \vp_{\grg}
  \end{align*}
  であることから。
\end{rem}




\bfsubsection{定理 8.1 直後}
\barquo{
次の図形は可換である。(垂直の矢は自然な写像)しかも、$\wh{f}$はこの可換図形によって一意的に定まる。
\[
\xymatrix{
M \ar[r]^-f \ar[d] & N \ar[d] \\
\wh{M} \ar[r]^-{\wh{f}} & \wh{N}
}
\]
}
\begin{proof}
  一意性は$M \to \wh{M}$の像の稠密性と、完備性(分離性)よりあきらか。可換性を示そう。
  まず、$\wh{f}$の定義から、
  \[
\xymatrix{
\wh{M} \ar[r]^-{\wh{f}} \ar[d]_-{\vp_{\grg}} & \wh{N} \ar[dl]^{p_{\grg}} \\
N/N_{\grg} & {}
}
  \]
  が可換であることに注意する。さらに、次の図式も可換である。
  \[
  \xymatrix{
  \wh{M} \ar[dr]_-{p_{\grl}} & M \ar[l]_-{\psi_M} \ar[d]^-{\pi_{\grl}} \ar[r]^-f & N \ar[d]^-{\pi_{\grg}} \ar[r]^-{\psi_N} & \wh{N} \ar[dl]^-{p_{\grg}} \\
  {} & M/M_{\grl} \ar[r]^-{f_{\grl \grg}} & N/N_{\grg} & {}
  }
  \]
  \[
  (M_{\grl} \subset f^{-1}(N_{\grg}))
  \]
したがって
\begin{align*}
  p_{\grg} \wh{f} \psi_M &= \vp_{\grg} \psi_M \\
  &= f_{\grl \grg}  p_{\grl} \psi_M  &(M_{\grl} \subset f^{-1}(N_{\grg})) \\
  &= f_{\grl \grg} \pi_{\grl} \\
  &= \pi_{\grg} f \\
  &= p_{\grg} \psi_N f
\end{align*}
である。ゆえに$\wh{N}$の普遍性(射の一意性)により
\[
\wh{f} \psi_M = \psi_N f
\]
が成り立つことがわかる。
\end{proof}



\bfsubsection{定理 8.2 直前}
\barquo{
容易にわかるように、$M$が$I$進位相で完備ということは、$M$の元の列$x_1, x_2, \cdots $が$x_i - x_{i+1} \in I^iM \; (\forall i)$をみたすとき、$x - x_i \in I^iM \; (\forall i)$をみたす$x \in M$が1つかつただ1つ存在することと同値である。
}
\begin{proof}
  ($\Rightarrow$) 完備性を仮定し、$M$の元の列$x_1, x_2, \cdots $が$x_i - x_{i+1} \in I^iM \; (\forall i)$をみたすとする。このとき$\wt{x} = (\ol{x_i})_{i \geq 1}$は$\wh{M}$の元である。完備性により$\psi \colon M \to \wh{M}$は同形なので、$\wt{x} = \psi(x) $なる$x \in M$
  がただ一つある。ここで
  \[
  \wt{x} = \psi(x) \iff x - x_i \in I^iM \; (\forall i)
  \]
  だから示すべきことがいえた。

  ($\Leftarrow$) 逆をいおう。一意性についての仮定から、$M$は分離的でなくてはならない。よって$\psi$は単射。また任意に$y =(y_i) \in \wh{M}$が与えられたとする。$y_i \in M/I^iM$の$M$への持ち上げを$x_i$とすると、$x_i - x_{i+1} \in I^iM$だから、仮定により$x - x_i \in I^iM \; (\forall i)$なる$x \in M$がある。このとき$\psi(x) =y$である。すなわち、$\psi$は全射。
\end{proof}



\bfsubsection{定理 8.3}
\barquo{
\[
F - G_n H_n = \sum \gro_i U_i(X), \quad \gro_i \in \frakm^n, \quad \deg U_i < \deg F
\]
}
\begin{rem}
  $G_n$, $H_n$, $F$はモニックなので$\deg (F - G_n H_n) < \deg F$がいえる。
\end{rem}



\bfsubsection{定理 8.3}
\barquo{
よって
\[
\deg w_i < \deg g
\]
である。
}
\begin{rem}
  $\deg v_i < \deg h$と$\deg w_i < \deg g$は、$G_{n+1}, H_{n+1}$のモニック性を保証するために必要。
\end{rem}





\bfsubsection{定理 8.7}
\barquo{
$A$をネーター環、$I$をイデアル、$M$を有限$A$加群とする。$M$, $A$の$I$進完備化を$\wh{M}$, $\wh{A}$とすれば
\[
M \ts_A \wh{A} \cong \wh{M}
\]
である。
}
\begin{proof}
  いったん$M$は任意の$A$加群であるとしておく。逆系$\cala \colon \N^{op} \to \Mod{A}$を
  \begin{gather*}
    \cala(n) = A/I^n  \\
    \cala(n \leq m) \colon A/I^m \to A/I^n
  \end{gather*}
  により定める。このとき逆系$M \ts \cala \colon \N^{op} \to \Mod{A}$が誘導される。このとき射影$\pi \colon \llim \cala \to \cala$ (ただし$\llim$は定数関手) は自然変換$M \ts \pi \colon M \ts \llim \cala  \to M \ts \cala $をひきおこす。したがって逆極限の普遍性により次の図式
  \[
  \xymatrix{
  \llim (M \ts \cala) \ar[r]^-p & M \ts \cala \\
  M \ts \llim \cala \ar[u]^-{\vp_M} \ar[ur]_{M \ts \pi} & {}
  }
  \]
  を可換にする射$\vp_M \colon M \ts \llim \cala \to \llim (M \ts \cala)$がある。このとき$\vp \colon \square \ts \llim \cala \to \llim (\square \ts \cala)$は自然変換になっていることを見よう。

  任意に$A$-線形写像$f \colon M_1 \to M_2$が与えられたとする。このとき自動的に$f$は$I$進位相で連続になることに気をつける。示すべきことは次の図式の可換性である。
  \[
  \xymatrix{
  M_1 \ts \llim \cala \ar[r]^-{f \ts \llim \cala} \ar[d]_-{\vp_{M_1}} & M_2 \ts \llim \cala \ar[d]^-{\vp_{M_2}} \\
  \llim (M_1 \ts \cala) \ar[r]^-{\la{f \ts \cala}} & \llim (M_2 \ts \cala)
  }
  \]
  生成元をとってくると
  \begin{align*}
    \la{f \ts \cala} \circ \vp_{M_1} (x \ts (a_i)_{i \in \N}) &= \la{f \ts \cala} ( (x \ts a_i)_{i \in \N}) \\
    &= (f (x) \ts a_i)_{i \in \N} \\
    \vp_{M_2} \circ f \ts \llim \cala (x \ts (a_i)_{i \in \N}) &= \vp_{M_2} (f(x) \ts (a_i)_{i \in \N}) \\
    &= (f(x) \ts a_i)_{i \in \N}
  \end{align*}
  が成り立つ。よって自然性がいえた。

  さてここで$M$が有限生成かつ$A$がNoetherという仮定を使おう。これにより、有限生成自由加群$F_1$, $F_2$による完全系列
  \[
  F_1 \to F_2 \to M \to 0
  \]
  が存在することがわかる。このとき$\vp$の自然性から次は可換である。
  \[
  \xymatrix{
  F_1 \ts \llim \cala  \ar[r] \ar[d]_-{\vp_{F_1}} & F_2 \ts \llim \cala \ar[r] \ar[d]_-{\vp_{F_2}} & M \ts \llim \cala \ar[r] \ar[d]_-{\vp_{M}} & 0 \\
  \llim (F_1 \ts \cala) \ar[r] & \llim (F_2 \ts \cala) \ar[r] & \llim (M \ts \cala) \ar[r] & 0
  }
  \]
  さらに上の行はテンソル積の右完全性から完全で、下の行は定理1と定理6により完全である。(定理1では特殊な形の短完全列しか扱っていないように見えるが、すべての短完全列は$0 \to N \to M \to M/N \to 0$という形の短完全列と同型であるので、実はそれで十分である) 有限生成自由加群についてはあきらかに同型がいえる。したがって、5項補題により$\vp_M$は同型。

\end{proof}


\bfsubsection{定理 8.10}
\barquo{
ii) $A$がネータ整域、$I$が$A$の真のイデアルならば
\[
\bigcap_{n>0} I^n = (0).
\]
}
\begin{rem}
  $A$が整域という仮定は必要である。たとえば次のような例がある。$K$を体とし、$A = K \tm K$, $I = K \tm \{0\}$とする。このとき$A$はネーター環だが任意の$n$について$I^n = I$である。
\end{rem}


\bfsubsection{定理 8.11}
\barquo{
$M/I^nM$は$I$進位相でディスクリート、したがって完備であって、
}
\begin{proof}
  実際に計算してみると$(M/I^n M)$の$I$進完備化は
  \begin{align*}
     \llim_{m} (M/I^n M)/ I^m(M/I^n M) &= \llim_m (M/I^n M)/(I^m M + I^n M / I^n M) \\
     &= \llim_m M/(I^m M + I^n M) \\
     &\cong M/I^n M
  \end{align*}
  である。
\end{proof}




\bfsubsection{定理 8.12}
\barquo{
前定理の証明で、$I^n\wh{M}$が$\wh{M} \to (M/I^nM)^{\wedge}$の核であることを示したのと同様にして、$J\wh{M}$は$\wh{M} \to (M/JM)^{\wedge}$の核であることがわかる。それは定理1より$\psi(JM)$の閉包に等しいから第2の等号が成立つ。
}
\begin{rem}
  正直に白状しますが、同様にしてみたけどわかりませんでした。代わりに次のようにして第2の等号を示した。
  \begin{align*}
    (JM)^{\wedge} &= JM \ts \wh{A} &(\text{$A$はNoether}) \\
    &= \Im(J \ts M \to M) \ts \wh{A} \\
    &= \Im (J \ts M \ts \wh{A} \to M \ts \wh{A}) &(\text{$\wh{A}$は$A$平坦}) \\
    &= \Im (J \ts \wh{M}  \to \wh{M}) \\
    &= J \wh{M}
  \end{align*}
\end{rem}



\bfsubsection{定理 8.14}
\barquo{
しかるに$\frakm$は$A$で閉集合であるから$\frakm \wh{A} \cap A = \frakm$が成り立つ。
}
\begin{proof}
  定理8.11により自然な写像$\psi \colon A \to \wh{A}$は像への同相なので$\frakm \clsub A$より$\psi(\frakm )\clsub \psi(A)$である。したがって$\psi(A)$において閉包をとっても変わらないのだから$\frakm \wh{A} \cap \psi(A) = \psi(\frakm)$である。したがって、$\psi$で引き戻して
  $\frakm \wh{A} \cap A = \frakm$がわかる。
\end{proof}




\bfsubsection{定理 8.15 直後}
\barquo{
4) $\wh{A}$もネータ局所環で、その極大イデアルは$\wh{A}$
}
\begin{rem}
$\wh{A}$が局所環になることは、次の補題による。
\end{rem}

\lem{
$A$は環、$\frakm \subset A$はイデアルで、$A$の$\frakm$進完備化を$\wh{A}$と書くことにする。$a=(a_i ) \in \wh{A}$とする。このとき次は同値。
\begin{description}
  \item[(1)] $a \in \wh{A}$は単元
  \item[(2)] $a_1 \in A/\frakm$は単元
\end{description}
}
\begin{proof}
  証明はやさしいが、書くのが面倒なので[雪江]\cite{雪江3} 定理3.1.13(2)を参照のこと。
\end{proof}


\newpage
\bfsection{\S 9 整拡大}



\bfsubsection{定理 9.1 直後}
\barquo{
とくに$A$が整域で、かつその商体の中で整閉であるとき、単に$A$が整閉整域であるという。環$A$の各素イデアル$\frakp$に対し$A_{\frakp}$が整閉整域であるとき、$A$を正規環とよぶ。
}
\begin{rem}
  次の命題が成り立つことが知られている。
\end{rem}

\prop{ (整拡大はテンソルで保たれる) \\
$k$は環、$A,B,C$は$k$代数であるような可換環とする。このとき$B$が$A$上整なら、$B \ts C$は$A \ts C$上整。
}
\begin{proof}
  生成元$b \ts c \in B \ts C$をとる。このとき、$b$は$A$上整なので
  \[
  b^n + a_{n-1} b^{n-1} + \cdots + a_{1} b + a_0 = 0
  \]
  なる$a_0 , \cdots , a_{n-1} \in A$がある。したがって
  \[
  (b \ts c)^n + (a_{n-1} \ts c)(b \ts c)^{n-1} + \cdots + (a_1 \ts c^{n-1}) (b \ts c) + (a_0 \ts c^n) = 0
  \]
  である。よって$b \ts c$は$A \ts C$上整。ゆえに、生成元がすべて整なので$B \ts C$は$A \ts C$上整。
\end{proof}


\prop{
(整閉包は局所化で不変) \\
$B$は$A$代数であり、$C$は$B$における$A$の整閉包であるとする。$S \subset A$は積閉集合とする。このとき$B_S$における$A_S$の整閉包は$C_S$に等しい。
}
\begin{proof}
  $C_S$が$A_S$上整であることは前の命題からあきらか。逆に$B_S$の元$b/s$が$A_S$上整だったとして$b/s \in C_S$を示そう。仮定より$b/s$は$A_S$上整であるが、$s$のべきを乗ずることにより$b$は$A_S$上整としてよい。したがって、
  \[
  b^n + \f{a_{n-1}}{t} b^{n-1}  + \cdots + \f{a_1}{t} b + \f{a_0}{t} = 0
  \]
  なる$a_0 , \cdots , a_{n-1} \in A$と$t \in S$がある。両辺に$t^n$をかけることにより、
  \[
    (tb)^n + a_{n-1} (tb)^{n-1}  + \cdots + a_1 t^{n-2} (tb) + a_0 t^{n-1} = 0
  \]
  をうる。よって$tb$は$A$上整だから、$tb \in C$である。すなわち$b/s \in C_S$がわかる。
\end{proof}



\prop{
$A$は整域とする。このとき次は同値。
\begin{description}
  \item[(1)] $A$は整閉整域
  \item[(2)] $A$は正規環
  \item[(3)] 任意の極大イデアル$\frakm \subset A$に対して$A_{\frakm}$は整閉整域
\end{description}
}
\begin{proof}
  証明はしなくてもいいという読者も多いだろう。が、念のために書いておく。
  \begin{description}
    \item[(1)$\To$(2)] 整閉包は局所化で不変であることによる。
    \item[(2)$\To$(3)] 自明。
    \item[(3)$\To$(1)] 環拡大$B/A$があるとき、$B$における$A$の整閉包を$\ol{B/A}$で書くことにする。(ここだけの記号。正直あんまりよくない記号だと思う) 任意の極大イデアル$\frakm \subset A$について$A_{\frakm}$は整閉整域なので、$K$を$A$の商体とすると
    \begin{align*}
      A_{\frakm} &= \ol{K/ A_{\frakm}} \\
      &= \ol{K \ts A_{\frakm} / A_{\frakm}} \\
      &= \ol{K/A} \ts A_{\frakm}
    \end{align*}
    したがって、$A$加群として$(\ol{K/A} / A) \ts  A_{\frakm} = 0$である。任意の極大イデアルで局所化して$0$なので定理4.6により$\ol{K/A} =A$である。つまり$A$は整閉。
  \end{description}
\end{proof}


\bfsubsection{定理 9.1 直後}
\barquo{
$A$がネーター環で上の意味で正規なら、$\frakp_1, \cdots ,\frakp_r$を極小素イデアルの全体とすれば、どんな素イデアル$\frakp$についても$A_{\frakp}$が整域ということから$(0) = \frakp_1 \cap \cdots \cap \frakp_r$, $\frakp_i + \frakp_j = A \; (i \neq j)$, したがって$A = A/ \bigcap \frakp_i = A/\frakp_1 \tm \cdots \tm A/\frakp_r$となり、各$A/\frakp_i$
は整閉整域である。逆に有限個の整閉整域の直積は正規環である。
}
\begin{proof}
  Zornの補題を用いることにより、任意の環は極小素イデアルを持つことがいえる。ここでは特に$A$はNoetherなので、定理6.5より極小素イデアルは有限個である。そこでそれらを$\frakp_1, \cdots ,\frakp_r$とおくことができる。環$A$の巾零元の全体$\sqrt{0}$は、各$A_{\frakp}$が整域であることから
  \[
  \sqrt{0} \ts A_{\frakp} = 0
  \]
  を満たす。したがって、$\frakp$は$A$の任意の素イデアルだったから$\sqrt{0}=0$である。したがって、巾零元の全体とは素イデアルの共通部分のことだったから、$(0) = \frakp_1 \cap \cdots \cap \frakp_r$が成り立つ。

  異なる$\frakp_i$が互いに素であること: ある$i,j$について$\frakp_i + \frakp_j \subset \frakm$なる極大イデアル$\frakm$が存在したとして、$\frakp_i = \frakp_j$を示せばよい。このとき、$\frakp_i \subset \frakm$かつ$\frakp_j \subset \frakm$なので、イデアル$\frakp_i A_{\frakm}$, $\frakp_j A_{\frakm}$は素イデアルである。$\frakp_i$, $\frakp_j$は極小素イデアルだったから、$\frakp_i A_{\frakm}$, $\frakp_j A_{\frakm}$
  も極小素イデアル。ところが仮定により$A_{\frakm}$は整閉整域、とくに整域なので、$A_{\frakm}$の極小素イデアルは$0$だけである。ゆえに$\frakp_i A_{\frakm}=\frakp_j A_{\frakm}$である。引き戻して$\frakp_i = \frakp_j$を得る。

 $A/\frakp_i$が整閉整域であること: 整域であることはあきらかなので、正規環であることを示せば十分である。$\frakq$を$A/\frakp_i$の素イデアルとする。$\frakq$は$\frakp_i$を含む$A$の素イデアルと同一視できて、
 \[
 (A/\frakp_i)_{\frakq} = A_{\frakq} / \frakp_i A_{\frakq}
 \]
 である。ここで$\frakp_i$の極小性により、$ \frakp_i A_{\frakq}$は極小素イデアル。ところが$A_{\frakq}$は仮定により整域なので、$\frakp_i A_{\frakq} = 0$でなくてはならない。よって$(A/\frakp_i)_{\frakq} = A_{\frakq}$となり、右辺は整閉整域だったから、$A/\frakp_i$が正規環であることがいえた。

 有限個の整閉整域の直積が正規環であること: $A$が整閉整域、$B$は$A$の有限個の直積$A \tm \cdots \tm A$とする。このとき、$\frakq \subset B$を素イデアルとすると、$\frakq$はある素イデアル$\frakp \subset A$に関して$A \tm \cdots \tm \frakp \tm \cdots \tm A$という形をしている。したがって$B_{\frakq} = (A \tm \cdots \tm A)_{\frakq} \cong A_{\frakp}$
 であり、$A$は整閉整域だから$A_{\frakp}$は整閉整域。したがって$B$は正規環である。
\end{proof}


\newpage
\bfsection{\S 10 一般付値}

\bfsubsection{\S 10 冒頭}
\barquo{
整域$R$が付値環であるとは、その商体$K$の各元$x$について
\[
x \not\in R \To x^{-1} \in R
\]
が成り立つことをいう。
}
\begin{rem}
  付値環の例として、たとえばべき級数環$R = \Q[[x]]$がある。これが付値環であることの証明は省略する。
\end{rem}


\bfsubsection{定理 10.2}
\barquo{
$A$を$A_{\frakp}$でおきかえて、$A$が局所環で$\frakp = \frakm_A$としてよい。
}
\begin{proof}
  書いてしまえばほとんど当たり前だが、一応説明しておく。局所環の場合に示せたとする。このとき、$A_{\frakp}$は局所環なので
  \[
  R \supset A_{\frakp}, \quad \frakm_{R} \cap A_{\frakp} = \frakp A_{\frakp}
  \]
  なる$K$の付値環$R$が存在する。このとき$R \supset A$であり、かつ
  \[
  \frakm_{R} \cap A = (\frakm_{R} \cap A_{\frakp}) \cap A = \frakp A_{\frakp} \cap A = \frakp
  \]
  が成り立つ。よって$A$について示すべきことがいえた。
\end{proof}




\bfsubsection{定理 10.4}
\barquo{
$K$の付値環で$A$を含むものたちの共通部分を$B'$とすれば、前定理により$B \subset B'$である。
}
\begin{proof}
  $x \in B$とし、$K$の付値環$R$で$A$を含むものがあたえられたとする。このとき$x$は$A$上整なので$R$上でも整である。したがって、付値環は整閉なので$x \in R$である。よって$B \subset R$であり、$R$は任意だったから$B \subset B'$である。
\end{proof}


\bfsubsection{定理 10.5}
\barquo{
すると$\scra$は次の性質をもつことが容易にわかる。
\begin{description}
  \item[$\gra$)] $F_1 ,\cdots , F_{r} \in \scra \; \Rightarrow \; F_1 \cap \cdots \cap F_{r} \in \scra$
  \item[$\beta$)] $Z_1 , \cdots , Z_n$が閉集合で$Z_1 \cup \cdots \cup Z_n \in \scra \; \Rightarrow \; \exists i : Z_i \in \scra$
  \item[$\grg$)] 閉集合$F$が$\scra$の元を含めば$F \in \scra$
\end{description}
}
\begin{proof} ${}$
  \begin{description}
    \item[$\gra$)] $\scra \cup \{F_1 \cap \cdots \cap F_{r}\}$は有限交性をもつので、$\scra$の極大性から。
    \item[$\beta$)] 有限個の閉集合$Z_1 , \cdots , Z_n$が与えられたとし、任意の$i$について$Z_i \not\in \scra$だとする。$\scra$の極大性により、$\scra \cup \{Z_i \}$は有限交性をもたない。したがって$\gra$)により$Z_i \cap F_i = \emptyset $なる$F_i \in \scra$が$i$ごとに存在する。このとき$(Z_1 \cup \cdots \cup Z_n) \cap \bigcap_{i=1}^n F_i = \emptyset$
    である。ゆえに有限交性がなくなるので$Z_1 \cup \cdots \cup Z_n \not\in \scra$である。
    \item[$\grg$)] $\scra \cup \{ F \}$は有限交性をもつので、$\scra$の極大性から。
  \end{description}
\end{proof}




\bfsubsection{定理 10.5}
\barquo{
補集合を表すには肩に$c$をつけることにすると、$F \in \scra$, $F^c = \bigcup_{\grl} U_{\grl}$なら$F = \bigcap_{\grl} U_{\grl}^c$であり、また$U(x_1, \cdots , x_n)^c = \bigcap_{i=1}^n U(x_i)^c \in \scra$なら上の$\beta)$によりどれかの$U(x_i)^c$が$\scra$に属する。よって$\scra$の元の共通部分は$\scra$に属する$U(x)^c$の形の集合の共通部分に等しい。
\[
\grG = \setmid{y \in K}{U(y^{-1})^c \in \scra }
\]
とおく。$R \in \Zar (K,A)$が$U(y^{-1})^c$に属するための条件は$y \in \frakm_R$であるから
\[
\text{$\scra$のすべての元の共通部分} = \setmid{R \in \Zar(K,A)}{\frakm_R \supset \grG}
\]
である。
}
\begin{rem}
  いろいろ書いてあるが、証明で必要なのは
  \[
  \text{$\scra$のすべての元の共通部分} \supset \setmid{R \in \Zar(K,A)}{\frakm_R \supset \grG}
  \]
  だけである。他は無視してもよい。そこで、これだけを示すことにする。簡単に示せる次の補題に気をつける。
\end{rem}

\lem{
$\grd$) \\
$Z_{\grl} \; (\grl \in \grL)$が閉部分集合で$\bigcap_{\grl} Z_{\grl} \in \scra$ならば、すべての$\grl \in \grL$について$Z_{\grl} \in \scra$である。
}
\begin{proof}
  $\grg$)によりあきらか。
\end{proof}

\begin{proof}
  引用部の証明に戻る。$\frakm_R \supset \grG$なる付値環$R \in \Zar(K,A)$と$F \in \scra$が任意に与えられたとする。補集合は$\Zar(K,A)$を全体集合としてとると約束しよう。このときZariski位相の定義により
  \[
  F^c = \bigcup_{\grl \in \grL} U_{\grl}
  \]
  なる開基$\scrf$の元$U_{\grl} $がある。このとき$\grl$ごとに有限集合$B(\grl)$が存在して
  \[
  U_{\grl}^c = \bigcup_{\tau \in B(\grl)} U(x_{\tau})^c
  \]
  が成り立つ。したがって
  \begin{align*}
    F &= \bigcap_{\grl \in \grL} U_{\grl}^c \\
    &= \bigcap_{\grl \in \grL} \bigcup_{\tau \in B(\grl)} U(x_{\tau})^c
  \end{align*}
  が成り立つ。$F \in \scra$と補題$\grd$)により、任意の$\grl$に対して
  \[
  \bigcup_{\tau \in B(\grl)} U(x_{\tau})^c \in \scra
  \]
  である。$\beta$)により、$\grl$ごとにある$\tau(\grl) \in B(\grl)$が存在して$U(x_{\tau(\grl)})^c \in \scra$である。すると
  \[
  F \supset \bigcap_{\grl \in \grL} U(x_{\tau(\grl)})^c
  \]
  となる。ここで$x_{\tau(\grl)}^{-1} \in \grG$より仮定から$x_{\tau(\grl)}^{-1} \in \frakm_R$となる。よって$x_{\tau(\grl)} \not\in R$なので$R \in U(x_{\tau(\grl)})^c $が結論される。$\grl$は任意だったから
  \[
  R \in  \bigcap_{\grl \in \grL} U(x_{\tau(\grl)})^c \subset F
  \]
  であって、これで示すべきことがいえた。
\end{proof}


\bfsubsection{定理 10.5 直後}
\barquo{
$G$は包含関係で全順序集合であるが、われわれは$G$に包含関係と逆の順序を入れることにする。
}
\begin{rem}
  包含関係と同じ順序をいれたとしても、順序群(あとで定義する)にはなっている。
\end{rem}


\bfsubsection{定理 10.5 直後}
\barquo{
公理から、
\[
(1) \; x>0,y \geq 0 \Rightarrow x+y > 0, \quad (2) \; x \geq y \Rightarrow -y \geq -x
\]
などが従う。
}
\begin{proof} ${}$
  \begin{description}
    \item[(1)] $x \geq 0$より$x + y \geq 0$である。もしも$x+y=0$なら、$x=-y$である。$-y \geq -y$と$y \geq 0$により$0 \geq -y$だから、$0 \geq x$となり、矛盾。よって$x+y>0$である。
    \item[(2)] $-x-y \geq -x-y$だから、両辺足して$-y \geq -x$を得る。
  \end{description}
\end{proof}


\bfsubsection{定理 10.5 直後}
\barquo{
$R_{\nu}$は$K$の付値環で$m_{\nu}$がその極大イデアルである。
}
\begin{proof}
  $R_{\nu}$が$K$の部分環 ($1,0 \in R$かつ$x,y \in R$なら$xy,x-y \in R$) であり、その商体が$K$で、付値環であるということ、そして$m_{\nu} \subset R_{\nu}$が全体ではないイデアルであるということはあきらか。極大イデアルであることを示そう。いま$x \in R_{\nu} \setminus m_{\nu}$とする。
  \[
  0 =\nu(1) = \nu (x x^{-1}) = \nu(x) + \nu(x^{-1}) = \nu(x^{-1})
  \]
  より$x^{-1} \in R_{\nu}$である。よって示せた。
\end{proof}



\bfsubsection{定理 10.5 直後}
\barquo{
対応$\nu \colon K \to G \cup \{\infty \}$を$\nu(0)=\infty$, $\nu(x)=xR \; (x \in K^*)$で定義すれば、$\nu$は$G$を値群とする加法付値になり
}
\begin{proof}
  加法付値の条件(1),(2),(3)を確認しよう。といっても(1)と(3)はあきらかなので、(2)だけを示す。$x,y,x+y$のいずれかが$0$なら証明することはないので、これらは$K^*$の元としてよい。$\nu(x) \geq \nu(y)$と仮定しよう。$xR \geq yR$より、順序の入れ方から$xR \subset yR$である。したがって$x = ry$なる$r \in R$がある。ゆえに$(x+y)R \subset yR$だから$\nu(x+y) \geq \nu(y)$である。よって示すべきことがいえた。
\end{proof}



\bfsubsection{定理 10.5 直前}
\barquo{
$H,H'$を値群とする$K$の二つの加法付値$\nu, \nu'$が共に$R$を付値環としてもてば、$H$から$H'$の上への順序を保つ同形写像$\vp$があって$\nu' = \vp \nu$が成り立つ(証明してみよ)。
}
\begin{proof}
  準同形定理により$K^* / R^{\tm} \cong H$かつ$K^* / R^{\tm} \cong H'$である。したがって次の図式
  \[
  \xymatrix{
  K^*/ R^{\tm} \ar[r]^{\ol{\nu} } \ar[dr]_{\ol{\nu'}} & H \ar[d]^{\vp} \\
  {} & H'
  }
  \]
を可換にするような群の同型$\vp$がある。$\vp$が順序を保つことをみよう。$x \in H$, $x \geq 0$とする。このとき$x = \nu(b)$なる$b \in R$がある。したがって
\begin{align*}
  \vp(x) &= \vp(\ol{\nu}(\ol{b}) ) \\
  &= \ol{\nu'}(\ol{b}) \\
  &= \nu'(b) \\
  &\geq 0
\end{align*}
である。
  \begin{comment}
  証明してみた。次の図式を可換にする$\vp$をみつければよい。
  \[
  \xymatrix{
  K^* \ar[r]^{\nu} \ar[dr]_{\nu'} & H \ar[d]^{\vp} \\
  {} & H'
  }
  \]
  $x \in H$とすると$x=\nu(a)$なる$a \in K^*$があるので、このとき
  \[
  \vp(x)=\nu'(a)
  \]
  と定めれば(well-definedならば)可換にできる。well-definedであることを示そう。$x = \nu(a) = \nu(b)$なる$a,b \in K^*$があったとする。このとき$ab^{-1} \in \Ker \nu$である。$R$は付値環なので、$ab^{-1}$と$a^{-1}b$のどちらかは$R$の元であり、したがって$ab^{-1}$は$R$の単元である。ゆえに$\nu'(ab^{-1})=0$であって、$\nu'(a)= \nu'(b)$となる。よって$\vp$はwell-defindである。

  $x = \nu(a)$かつ$y = \nu(b)$だとする。このとき
  \begin{align*}
    \vp(x+y) &= \vp(\nu(a)+\nu(b) ) \\
    &= \vp(\nu(ab)) \\
    &= \nu'(ab) \\
    &= \nu'(a) + \nu'(b) \\
    &= \vp(x) + \vp(y)
  \end{align*}
  だから$\vp$は準同形。

  もし$x \geq 0$ならば、$x = \nu(a)$なる$a$は$R$の元である。よって$\vp(x)=\nu'(a) \geq 0$である。つまり$\vp$は順序を保つ。
\end{comment}
\end{proof}



\bfsubsection{定理 10.6}
\barquo{
任意の$y \in G$に対し、$nx \leq y$となる$n \in \Z$の中で最大のものが定まる。
}
\begin{proof}
  仮定により$n' x > y$なる$n'$があり、$n'$より大きい$n$についても$nx>y$なので、集合$B(y) = \setmid{n \in \Z}{nx \leq y}$は上に有界である。集合$B(y)$が空でないことは、$y \geq 0$のときは$0 \in B(y)$よりわかる。よって、$y \geq 0$ならば$B(y)$は最大元を持つ。$y < 0$とする。このとき、$B(-y)$は最大元を持つので、
  \[
  nx \leq -y < (n+1)x
  \]
  となるような$n$がある。このとき$-(n+1) \in B(y)$である。したがって空でないので、$B(y) \; (y<0)$も最大元を持つ。
\end{proof}


\bfsubsection{定理 10.6}
\barquo{
$y_1 = y - n_0 x$とおき$nx \leq 10y_1$となる最大の整数$n$を$n_1$とおけば$0 \leq n_1 < 10$である。
}
\begin{proof}
  $0 \leq y_1$より、$n_1$の最大性から$0 \leq n_1$がわかる。また$10$を代入してみると
  \begin{align*}
    10y_1 - 10x &= 10y - 10n_0x - 10 x \\
    &= 10y - 10(n_0 + 1)x \\
    &< 0 &(n_0\text{の最大性から})
  \end{align*}
  が成り立つため、$n_1 < 10$である。
\end{proof}



\bfsubsection{定理 10.6}
\barquo{
$\vp \colon G \to \R$が順序集合として準同形、すなわち$y < y'$ならば$\vp(y) \leq \vp(y')$であることが容易にたしかめられる。
}
\begin{proof}
  $y < z$と仮定する。
  \begin{gather*}
    n_0 = \max\setmid{n}{nx \leq y} \\
    m_0 = \max\setmid{m}{mx \leq z}
  \end{gather*}
  とすると$n_0 \leq m_0$である。$n_0 < m_0$なら示すべきことはないので$n_0 = m_0$とする。このとき$y_1 = y - n_0x$, $z_1 =z - m_0 x $とすると$y_1 < z_1$であって、
  \begin{gather*}
    n_1 = \max\setmid{n}{nx \leq 10y_1} \\
    m_1 = \max\setmid{m}{mx \leq 10z_1}
  \end{gather*}
  とすると$n_1 \leq m_1$である。$n_1 < m_1$なら示すべきことはないので$n_1 = m_1$とする。このとき$y_2 = 10y_1 - n_1x$, $z_2 =z_1 - m_1 x $とすると$y_2 < z_2$である。以下、繰り返しにより$n_i$, $m_i$はすべて一致するか、またはある$k$について$n_k < m_k$かつ、すべての$i<k$について$n_k = m_k$である。よって$\vp(y) \leq \vp(z)$がいえた。
\end{proof}


\bfsubsection{定理 10.6}
\barquo{
さらに$\vp$は単射である。それを見るには、$y < y'$なら$x < 10^r(y' - y)$となる自然数$r$が存在することを注意すればよい。(詳細は読者に委ねる)
}
\begin{rem}
  委ねられてしまった。次の補題を使う。
\end{rem}

\lem{
定理10.6の状況で考える。$r \geq 0$に対して
\[
N(r) = \sum_{i=0}^r 10^{r-i}n_i
\]
とおく。このとき
\[
N(r)x = 10^ry -y_{r+1}
\]
が成り立つ。
}
\begin{proof}
  証明は$r \geq 0$についての帰納法による。$r=0$のとき$N(r)=n_0$であり、$n_0x=y-y_1$だから成り立っている。$r \geq 1$とし、$r-1$までの成立を仮定する。このとき
  \begin{align*}
    N(r)x &= (10N(r-1) + n_r)x \\
    &= 10N(r-1)x + n_rx \\
    &= 10(10^{r-1}y -y_r) + n_rx &(\text{帰納法の仮定による}) \\
    &= 10^r y - 10 y_r + n_r x \\
    &= 10^r y - y_{r+1} &(r \geq 1\text{より})
  \end{align*}
  だから、示すべきことがいえた。
\end{proof}

\lem{
$r \geq 1$のとき$0 \leq  y_r < x $が成り立つ。
}
\begin{proof}
  $n_0 x \leq y < (n_0 + 1)x$により$0 \leq y - n_0 x < x$である。よって$0 \leq y_1 < x$が成りたつ。また$k \geq 1$のとき$n_k x \leq 10y_k < (n_k + 1)x$であるので、$0 \leq 10y_k - n_kx  < x$がわかる。したがって$0 \leq y_{k+1} < x$である。
\end{proof}

\lem{
$n \in \Z$, $r \geq 0$, $y \in G$とする。次は同値。
\begin{description}
  \item[(1)] $n = N(r)$である。とくに$\vp(y)$の小数第$r$桁までの表示であって、$\f{n}{10^r}$と等しいものがある。(一般に小数展開は一意ではないことに注意)
  \item[(2)] $nx \leq 10^r y < (n+1)x$が成り立つ。
\end{description}
}
\begin{proof}
  ${}$
  \begin{description}
    \item[(1)$\To$(2)] $10^r y - x < nx \leq 10^r y$を示せばよいが、$0 \leq y_{r+1} < x$と$N(r)x = 10^r y - y_{r+1}$であることからそれはあきらか。
    \item[(2)$\To$(1)] (2)の両辺から$y_{r+1}$を引いて
    \[
    nx - y_{r+1} \leq N(r)x < (n+1)x - y_{r+1}
    \]
    である。よって
    \[
    (n - N(r))x \leq y_{r+1} < (n+1 - N(r))x
    \]
    がわかる。ここで$0 \leq y_{r+1} < x$により、$-1 < n - N(r) < 1$でなくてはならない。$n - N(r) \in \Z$だから、$n=N(r)$でしかありえない。
  \end{description}
\end{proof}

\begin{proof}
引用部の証明に戻る。$\vp(y) = \vp(y')$であったとしよう。このときすべての$r \geq 0$に対して$N(r) = N'(r)$であるので、
\begin{align*}
  N(r)x \leq &10^r y < (N(r)+1)x \\
  N(r)x \leq &10^r y' < (N(r)+1)x
\end{align*}
である。したがって
\begin{align*}
  &10^r(y - y') \leq (N(r)+1)x - N(r)x = x \\
  &10^r(y' - y) \leq (N(r)+1)x - N(r)x = x
\end{align*}
が成り立つ。ここでもしも$y \neq y'$ならば、仮定により十分大きな$r$についてどちらかは成り立たないはずなので、矛盾。よって$y = y'$である。
\end{proof}


\bfsubsection{定理 10.6}
\barquo{
次に$\vp$が群の準同形であることを見よう。
}
\begin{proof}
  $y,y' \in G$をとったとする。$N'(r)$を$N(r)$と同様のものとする。このとき
  \begin{align*}
N(r)x \leq 10^r y < (N(r)+1)x \\
N'(r)x \leq 10^r y < (N'(r)+1)x
  \end{align*}
  である。両辺足して
  \[
(N(r)+N'(r))x \leq 10^r y < (N(r) + N'(r) + 2)x
  \]
  となる。したがって$M(r)=N(r)+N'(r)$とおくと、
  \[
  M(r)x \leq 10^r(y+y') < (M(r)+1)x
  \]
  かまたは
  \[
  (M(r)+1)x \leq 10^r(y+y') < (M(r)+2)x
  \]
  のどちらかが成り立つ。したがって、
  \[
  0 \leq \vp(y+y') - \f{M(r)}{10^r} < \f{1}{10^r}
  \]
  \[
  0 \leq \vp(y+y') - \f{M(r)+1}{10^r} < \f{1}{10^r}
  \]
  のどちらかは正しい。このときいずれにせよ
  \[
  0 \leq \vp(y+y') - \f{M(r)}{10^r} < \f{2}{10^r}
  \]
  である。ゆえに
  \begin{align*}
    \abs{\vp(y+y') - \vp(y) - \vp(y')} &\leq \abs{\vp(y+y') - \f{M(r)}{10^r}} + \abs{ \f{N(r)}{10^r} - \vp(y) } + \abs{ \f{N'(r)}{10^r} - \vp(y') } \\
    &\leq \f{2}{10^r} + \f{1}{10^r} + \f{1}{10^r} \\
    &\leq \f{4}{10^r}
  \end{align*}
  がわかる。$r$は任意だから、$\vp(y+y') = \vp(y) + \vp(y')$がいえた。
\end{proof}


\bfsubsection{定理 10.7}
\barquo{
付値環$R$の値群を$G$とすると
\[
G\text{が階数$1$} \iff R\text{のKrull次元が$1$}
\]
}
\begin{proof}
  $\To$は本文通り。$\Leftarrow$を示す。$\dim R = 1$より、$R$は体ではない。したがって$G = \nu(R \setminus \{0\} )$は$0$ではない。ゆえに、定理10.6により、$a,b \in G$であって$a>0$なるものが与えられたとして、$na > b$なる$n$が存在することをいえばよい。$b \leq 0$ならば示すことはないので、$b>0$としてよい。

  $\nu$を$R$に対応する加法付値とする。$a=\nu(\xi)$, $b=\nu(\eta)$を満たす$\eta , \xi \in \frakm_R \setminus \{0\}$がある。$\dim R = 1$であるという仮定と、$R$のイデアル全体は全順序集合をなすということから、$\frakm_R$は$\eta \frakm_R$を含む唯一の素イデアルである。ゆえに
  \[
  \frakm_R = \sqrt{\eta \frakm_R}
  \]
  であるので、$\xi^n \in \eta \frakm_R$なる$n$がある。これは$na - b > 0$を意味する。よって、$G$は$0$でないarchimedian順序群であり、階数$1$である。
\end{proof}


\newpage
\bfsection{\S 11. DVR, Dedekind 環}

\bfsubsection{定理 11.1}
\barquo{
$a,b,c,d \in R - \{0\}$, $a/b=c/d$ならば
\[
v(a) - v(b) = v(c) - v(d)
\]
となることが容易にわかるから、$K^*$の元$\xi = a/b$に対し$v(\xi) = v(a) - v(b) \in \Z$とおけば、$v$は$K$の加法付値を定めその付値環が$R$になることは見易い。
}
\begin{proof}
  直感的に成り立ちそうだということは判る。確認していく。
  $a/b=c/d$とする。$K$の部分$R$加群として$a/b R = c/d R$である。一方で$a/b R = x^{v(a)-v(b)}R$であり、$c/d R = x^{v(c)-v(d)}R$であるから、$v(a) - v(b) = v(c) - v(d)$となることがいえた。

$a,b \in R \setminus \{0\}$とする。$v(ab)=v(a) + v(b)$であることはあきらか。
よって$\xi , \tau \in K^{\tm}$に対して$\xi = a/b$, $\tau = c/d$とすると
    \begin{align*}
      v(\xi \tau ) &= v(ac/bd) \\
      &= v(ac) - v(bd) \\
      &= v(a) + v(c) - v(b) - v(d) \\
      &= v(\xi) + v(\tau)
    \end{align*}
    がわかる。

 $a,b \in  R \setminus \{0\}$について、$v(a) \geq v(b)$とする。$v(a+b) \geq v(b)$となることはあきらか。
    $\xi , \tau \in K^{\tm}$が与えられたとする。このとき$\xi = a/b$, $\tau = c/d$とすると
    \begin{align*}
      v(\xi + \tau ) &= v\left( \f{a}{b} + \f{c}{d} \right) \\
      &= v \left( \f{ad+bc}{bd} \right) \\
      &= v(ad+bc) - v(bd) \\
      &\geq \min\{v(ad), v(bc) \} - v(bd) \\
      &= \min\{v(a)+v(d),v(b ) + v(c) \} - v(b)-v(d) \\
      &= \min\{v(a)-v(b),v(c)-v(d) \} \\
      &= \min \{ v(\xi), v(\tau)\}
    \end{align*}
    がわかる。
\end{proof}


\bfsubsection{定理 11.1}
\barquo{
$v$の値群はあきらかに$\Z$であるから$R$はDVRである。
}
\begin{rem}
  \textblue{これは正しくない。}$R$がもし体なら、$v$の値群は$0$である。したがって、(2)と(3)には「$R$は体でない」という仮定が必要である。
\end{rem}


\bfsubsection{定理 11.2 直前}
\barquo{
付値環$S$の極大イデアル$\frakm_S$が単項イデアルであっても$S$がDVRであるとは限らない。
}
\begin{proof} ${}$
\begin{description}
  \item[Step 1] $k[x,y]$の拡大環$A=k[x,y, x/y, x/y^2 , x/y^3 , \cdots ]$を考える。$M=yA$とする。$M=(x,y , x/y, x/y^2 \cdots )A$とも書けるので$M \subset A$は極大イデアルである。そこで$S = A_M$, $\frakm = M A_M$とする。$\frakm$は$S$の極大イデアルであり、かつ単項生成である。
  \item[Step 2] $S$がDVRでないことを示そう。$n \geq 0$に対し
  \[
  M_n = (x,y,x/y , \cdots , x/y^n)A
  \]
  とする。$M_n \subset A$は素イデアルであり、任意の$n$について
  \[
  M_0 \subsetneq \cdots \subsetneq M_{n-1} \subsetneq M_n \subsetneq M
  \]
  を満たすので$\height M \geq n$である。したがって$\height \frakm = \height M = \infty$である。Noether環のすべての素イデアルは高さが有限なので、$S$はNoetherでない。とくに、$S$はDVRでない。
  \item[Step 3] $S$が付値環であることを示そう。まず次の補題を示す。
  \lem{
  任意の$f \in k[x,y]$に対して
  \[
  f= x^s y^t g
  \]
  なる$s,t \geq 0$と$g \in A \setminus M$が存在する。
  }
\begin{proof}
  $f \in k[x,y]$より、
\[
f = \sum_{(s,t) \in I} a_{s,t} x^s y^t
\]
と表せる。$a_{s,t} \neq 0$としてよい。$s_0 \geq 0$を$s_0 = \min \setmid{s}{(s,t) \in I}$で定める。そして$J \subset I$を$J = \setmid{(s,t) \in I}{s=s_0}$とおくと
  \[
  f x^{-s_0} = \sum_{(s,t) \in J} a_{s,t} y^t + \sum_{(s,t) \in I \setminus J} a_{s,t} x^{s-s_0} y^t
  \]
  を得る。さらに$t_0 = \min \setmid{t}{(s_0,t) \in J}$とおくと
  \[
  f x^{-s_0} y^{-t_0} = a_{s_0,t_0 } + \sum_{(s,t) \in J \setminus \{(s_0, t_0)\}} a_{s,t} y^{t-t_0} +  \sum_{(s,t) \in I \setminus J} a_{s,t} x^{s-s_0} y^{t-t_0}
  \]
  がわかる。ここで$(s,t) \in I \setminus J$のとき
  \[
   x^{s-s_0} y^{t-t_0} = \f{x}{y^{t_0 - t}} \cdot x^{s-s_0-1}
  \]
  なので、$s-s_0 + 1 \geq 0$により$ x^{s-s_0} y^{t-t_0} \in M$である。また、あきらかに$(s,t) \in J \setminus \{(s_0, t_0)\}$のとき$y^{t-t_0} \in M$である。ゆえに$f x^{-s_0} y^{-t_0} \in A \setminus M$である。
\end{proof}
$S$が付値環であることの証明に戻る。$S$の商体は$k(x,y)$である。$h \in k(x,y)$が与えられたとする。$h=f_1 / f_2$なる$f_i \in k[x,y]$をとり、さらに$f_i = x^{s_i} y^{t_i} g_i$なる$s_i, t_i$と$g_i \in A \setminus M$をとる。このとき
\begin{align*}
  h &= \f{x^{s_1} y^{t_1} g_1  }{ x^{s_2} y^{t_2} g_2 } \\
  &= \f{ x^{s_1 - s_2} }{ y^{t_2 - t_1} } \cdot \f{g_1}{g_2}
\end{align*}
である。よって$s_1 - s_2 > 0$ならば$h \in A_M = S$である。$s_1 - s_2 < 0$ならば、$h^{-1} \in S$である。また$s_1 = s_2$のとき
\[
h = \f{y^{t_1 - t_2} g_1}{g_2}
\]
である。よって$t_1 \geq t_2$ならば$h \in S$である。$t_1 < t_2$ならば、$h^{-1} \in S$である。以上の議論により、$S$は付値環である。
\end{description}
\end{proof}



\bfsubsection{定理 11.2 (3)$\To$(1)}
\barquo{
また$x$がべき零なら$\dim R = 0$となるから$x^{\nu} \neq 0 \; (\forall \nu)$.
}
\begin{proof}
  $x$がべき零とする。このとき$\frakm$はべき零で、したがって
  \[
  \frakm \subset \sqrt{0} = \bigcap_{\frakp \in \Spec R} \frakp
  \]
  である。ゆえに、任意の素イデアル$\frakp$に対して$\frakm \subset \frakp$が成り立つが、$\frakm$は極大イデアルだったので$\frakm = \frakp$である。したがって$R$の素イデアルは$\frakm$だけなので、$\dim R = 0$である。
\end{proof}


\bfsubsection{定理 11.2 (3)$\To$(1)}
\barquo{
$v(t) = \nu$とおけば$v$は$R$の商体の加法付値で$R$がその付値環であることが容易にわかる。
}
\begin{proof}
  定理11.1の(2)$\To$(1)の証明と同様。
\end{proof}


\bfsubsection{定理 11.2 (1)$\To$(4)}
\barquo{
付値環だから正規である。
}
\begin{proof}
  一般に、付値環$R$は整域かつ整閉(定理10.3)なので整閉整域。したがって正規環である。
\end{proof}


\bfsubsection{定理 11.2 (4)$\To$(3)}
\barquo{
$R$は仮定により整域である。
}
\begin{proof}
  一般に、局所環$(R,\frakm)$が正規環でもあるなら、$R$は整域であることを示そう。$x,y \in R$とし、$xy=0$とする。準同形$\vp \colon R \to R_{\frakm}$で送ると、$R$が正規環という仮定より、$R_{\frakm}$は整閉整域なので$\vp(x)=0$または$\vp(y)=0$である。$\vp(x)=0$として一般性を失わない。このときある$u \in R \setminus \frakm$があって、$ux=0$である。ところが$u$は単元だったから、これは$x=0$を意味する。よって$R$
  は整域。
\end{proof}



\bfsubsection{定理 11.2 (4)$\To$(3)}
\barquo{
したがって定理8.10(i)により$\frakm \neq \frakm^2$であるから、
}
\begin{proof}
  NAKによっても示せる。$A$はNoethernなので$\frakm$は有限生成$A$加群である。もし$\frakm = \frakm^2$なら、NAKにより$a \frakm = 0$かつ$a \equiv 1 \mod \frakm$なる$a$の存在がわかる。$(A,\frakm)$は局所環なので$a$は単元で、したがって$\frakm = 0$となり矛盾。
\end{proof}


\bfsubsection{定理 11.2  (4)$\To$(3)}
\barquo{
$\dim R = 1$により$\frakm$は$xR$の素因子であり、$xR : y = \frakm$なる$y \in R$が存在する。
}
\begin{proof}
  $R$のイデアル$xR \subset R$の素因子とは、$R$加群$R/xR$の素因子を指すのだった。$xR \subset \frakm$より$R/ xR$は$0$でない$R$加群で、$R$はNoetherだったので定理6.1により$\Ass (R/xR) \neq \emptyset$である。そこで$P \in \Ass (R/xR)$とし、$P = \ann_{R}(\ol{y})$となる$\ol{y} \in R/xR$をとる。$\frakm \supset P \supset xR \supset (0)$である。
  $\dim R = 1$より$\frakm = P$でなくてはならない。したがって$\frakm$はイデアル$xR \subset R$の素因子である。$\ol{y}$の$R$における代表元のひとつを$y \in R$とすれば、これは$\frakm = xR : y$を意味する。
\end{proof}



\bfsubsection{定理 11.3 (2)$\To$(1)}
\barquo{
$I$から$R$への$R$線形写像はすべて$K$の元による乗法で得られる (証明せよ)。
}
\begin{proof}
  $I$が射影的という仮定は必要としないことを注意しておく。%$I$は分数イデアルなので、$\gra I \subset R$なる$\gra \in K^{\tm}$がある。$R$は整域なので$I \cong \gra I$であり、したがって$I \subset R$としてよい。
  また、分数イデアルの定義は$(0)$を除外しているので、$I \neq (0)$である。

局所化の平坦性により、$R$加群として$K$は平坦である。したがって、$I \ts K \subset K \ts K$だと見なせる。したがって、$K \ts K \cong K$より、$\dim_K I \ts K \leq 1$である。また、$I$は$0$でないので、$a \in I$なる$0$でない元$a$がある。$R$は整域なので、$aR$は自由加群である。ゆえに$K \cong aR \ts K \subset I \ts K$が判る。
したがって$\dim_K I \ts K = 1$であり、包含関係があって次元が同じなので$I \ts K = K \ts K$が結論できる。

ここで、$\vp \colon I \to R$が与えられたとする。
\[
\xymatrix{
K \ts K \ar@{=}[r] \ar[d]_-i & I \ts K \ar[r]^-{\vp \ts K} & R \ts K \ar[d]^-i \\
K \ar[rr] & {} & K
}
\]
あきらかに、$i \circ (\vp \ts K) \circ i^{-1} (x) = \beta x$なる$\beta \in K$がある。したがって、
\begin{align*}
  \vp(y) &= i \circ (\vp \ts K) (y \ts 1) \\
  &= (i \circ (\vp \ts K) \circ i^{-1}) (y) \\
  &= \beta y
\end{align*}
が成り立つ。
\end{proof}

\begin{rem}
  より初等的と思われる別証明を紹介する。$x \in I$は$K$の元なので、$x = b/c \; (c \in R, b \in R)$と表せる。$I$は分数イデアルなので、$I \cap R$は$0$でない。そこで$0 \neq a \in I \cap R$をとれる。すると$ac \vp(x) = ac \vp(b/c) = \vp(ab) = b \vp(a)$となる。ゆえに$\vp(x) = b c^{-1} a^{-1}\vp(a) = a^{-1} \vp(a) x$が成り立つ。
\end{rem}


\bfsubsection{定理 11.3 (1)$\To$(3)}
\barquo{
$a_i \in I$, $b_i \in I^{-1}$とし$P$を任意の素イデアルとすると、少なくとも1つの$i$に対して$a_i b_i$が$R_P$の単元となり、そのとき$I_P = a_i R_P$となるから$I_P$は単項イデアルである。
}
\begin{proof}
  $a_ib_i$は単元なので、$a_ib_i u = 1$なる$u \in R_P$がある。このとき任意の$j$について$a_j = (a_jb_i) u a_i $である。$b_i \in I^{-1}$により$a_j b_i \in R$であるから、$a_j \in a_i R_P$がわかる。$a_j$は$R$加群として$I$を生成するので、したがって$I \subset a_i R_P$が結論できる。逆は明らかなので、$I = a_i R_P$である。
  \begin{comment}
  すでに、(1)$\To$(2)は示したので、$I$は$R$加群として射影的である。局所環上の射影加群は自由加群である(定理2.5)ことに帰着させる方針で示す。
  $I$は分数イデアルなので、$I$は$R$のあるイデアルと同型であり、はじめから$I \subset R$としてよい。まず、局所化によって射影的という性質は不変であることをみる。
  \lem{
  $R$は環、$S$は$R$の積閉集合で、$P$は射影的$R$加群であるとする。このとき局所化$P_S$は$R_S$上の射影加群である。
  }
\begin{proof}
$R_S$加群$M,N$と$R_S$準同形$\vp \colon M \to N$, $\psi \colon P_S \to N$が与えられ、$\vp$は全射であるとする。$P$は射影的という仮定から、$R_S$加群を$R$加群だと思えば、次の図式
\[
\xymatrix{
{} & P \ar[d] \ar[ddl]_-{f} \\
{} & P_S \ar[d]^-{\psi} \\
M \ar[r]^-{\vp} & N \ar[r] & 0
}
\]
が可換になるような$R$準同形$f$の存在がわかる。全体を$R$加群だとみなし、$R_S$をテンソルする。すると、任意の$R_S$加群$C$について
\[
C \ts_R R_S = C \ts_{R_S} R_S \ts_R R_S = C \ts_{R_S} R_S = C
\]
が成り立つことから、(詳細な議論は省く) 次の可換図式を得る。
\[
\xymatrix{
{} & P_S \ar[d]^-{id} \ar[ddl]_-{f \ts R_S} \\
{} & P_S \ar[d]^-{\psi} \\
M \ar[r]^-{\vp} & N \ar[r] & 0
}
\]
以上により、$P_S$は$R_S$加群として射影的である。
\end{proof}
引用部の証明に戻る。$I$は射影$R$加群なので、$I \ts_R R_P$は射影$R_P$加群である。$R_P$は局所環であるため、$I \ts_R R_P$は自由$R_P$加群であることまでいえる。ここで、局所化の平坦性により$I \ts_R R_P = \Im(I \ts_R R_P \to R \ts_R R_P) = \Im(I \ts_R R_P \to R \ts_R R_P \to R_P) = I R_P = I_P$が保証されていることに気をつける。$I_P \subset R_P$であることから、$I_P$の$R_P$自由加群としての階数は$1$以下であることがすぐにわかる。
$I$は整域$R$の$0$でないイデアルなので、$I$はある$0$でない自由加群を含む。したがって$I_P$の階数は$1$であると決定する。したがって、$I_P \subset R_P$は単項イデアルである。
\end{comment}
\end{proof}


\bfsubsection{定理 11.3 (3)$\To$(1)}
\barquo{
$I$が有限生成なら$(I^{-1})_P = (I_P)^{-1}$である。
}
\begin{rem}
  $K$の部分加群として$I^{-1} = (R : I)$, $(I_P)^{-1} = (R_P : I_P)$と定義されていることにさえ注意すれば、本文通りの証明で示せる。
\end{rem}



\bfsubsection{定理 11.5}
\barquo{
$R = \bigcap_{ \mathrm{ht} P = 1} R_P$
}
\begin{proof}
$\subset$はあきらか。$a,b \in R$, $a \neq 0$とし、$\forall P \; \height P = 1 \To b/a \in R_P$と仮定する。$b/a \in R$を示したい。
$a \in R$が単元ならなにも示すことはないので、$a$は単元でないとしてよい。$R$はNoether環で、$aR \subsetneq R$なので、$aR$の準素分解
\[
aR = \frakq_1 \cap \cdots \cap \frakq_n, \quad \Ass(R / \frakq_i) = \{P_i\}
\]
が存在する。このとき$\frakq_i$は$R$の準素イデアルであり、$\sqrt{\frakq_i} =\sqrt{\ann(R/\frakq_i)} =  P_i$が成り立つ。

このとき任意の$i$について%$aR_{P_i} \cap R = \frakq_i$を示そう。まず
%\[
%\frakq_i \subset R, \quad \frakq_i \subset aR \subset aR_{P_i}
%\]
%により$aR_{P_i} \cap R \supset \frakq_i$はあきらか。逆に、
\begin{align*}
  aR_{P_i} &= (\frakq_1 \cap \cdots \cap \frakq_n )R_{P_i} \\
  &\subset \frakq_1 R_{P_i} \cap \cdots \cap \frakq_n R_{P_i} \\
  &\subset \frakq_i R_{P_i}
\end{align*}
である。したがって$aR_{P_i} \cap R \subset \frakq_i R_{\frakp_i} \cap R = \frakq_i$が成り立つ。

(i)により、単項イデアル$aR$の素因子$P_i$は高さ$1$である。したがって$b \in aR_{P_i} \cap R \subset \frakq_i$なので、$b \in \bigcap_{i} \frakq_i = aR$である。したがって$b/a \in R$が結論される。
\end{proof}



\bfsubsection{定理 11.6 (1)$\To$(3)}
\barquo{
$R$はすでにみたようにネータ環だから、$I$の大きさについての上からの帰納法が使える。
}
\begin{rem}
  $\scrs$を$R$のイデアル$J$であって、有限個の素イデアルの積として表せないもの全体の集合とする。$\scrs \neq \emptyset$と仮定すれば、$R$はNoetherなので$\scrs$は極大元$I$をもつ。あとは本文と同様に議論を進めれば、「上からの帰納法」とは何かという問題に触れずに済む。
\end{rem}


\newpage
\bfsection{\S 12 Krull 環}

\bfsubsection{\S 12 冒頭}
\barquo{
$A$がKrull環であるとは、$K$を商体とするDVRの族$\scrf = \{ R_{\grl } \}_{\grl \in \grL}$があって、$R_{\grl}$に対応する正規化された加法付値を$v_{\grl}$とするとき
\begin{description}
  \item[(1)] $A = \bigcap_{\grl} R_{\grl}$
  \item[(2)] $K^*$の各元$x$に対し、$v_{\grl}(x) \neq 0$となる$\grl \in \grL$は高々有限個である、
\end{description}
の$2$条件が成り立つことをいう。
}
\begin{rem}
  添え字集合$\grL$として空集合を許す。したがって、体は(DVRではないが)Krull環である。
\end{rem}

\bfsubsection{定理 12.1 直前}
\barquo{
DVRは完整閉であるからKrull環も完整閉である。
}
\begin{rem}
  完整閉という言葉は演習9.5で定義されている。DVRはNoether整閉整域なので、DVRが完整閉であることは演習9.4からあきらか。
\end{rem}


\bfsubsection{定理 12.1 直前}
\barquo{
$A$がKrull環なら、$K$の任意の部分体$K'$に対し$A \cap K'$もKrull環である。
}
\begin{proof}
$K' \subset A$のときにはあきらか。$K' \subset A$でないとして示す。Krull環$A$を定義するDVRの族$\scrf = \{ R_{\grl } \}_{\grl \in \grL}$をとり、$R_{\grl}$に対応する加法付値を$v_{\grl}$で表す。$v_{\grl} \colon K^{\tm} \to \Z$を$(K')^{\tm}$に制限したものを$v_{\grl}'$と書くことにする。
  $v_{\grl}'$は体$K'$の加法付値を定めており、その付値環は$R_{\grl } \cap K'$である。付値環$R_{\grl}$の極大イデアルを$\frakm_{\grl}$と表すことにする。このとき
  \begin{align*}
    K' \cap \frakm_{\grl} = (0) &\iff  (K')^{\tm} \subset K \sm \frakm_{\grl} \\
    &\iff (K')^{\tm} \subset R_{\grl}^{\tm} &(\text{$R_{\grl}$は付値環だから})\\
    &\iff K' \subset R_{\grl}
   \end{align*}
   であることに気をつける。

   $I = \setmid{\grl \in \grL}{K' \cap \frakm_{\grl} = (0)}$とし、$J = \grL \sm I$とする。補集合を$K$を全体として取ることにすると
   \begin{align*}
    K' \subset  A  &\iff K' \subset \bigcap_{\grl} R_{\grl} \\
    &\iff \forall \grl \quad K' \subset R_{\grl}
   \end{align*}
   だから、$K' \subset A$でないという仮定から、$J \neq \emptyset$がわかる。 $\grl \in J$のとき、$R_{\grl}$は体でなく、DVRである。また$\grl \in I$, $\mu \in J$のとき
   \[
   R_{\grl} \cap K' = K' \supset R_{\mu} \cap K'
   \]
   により、$A \cap K' = \bigcap_{\mu \in J} R_{\mu} \cap K'$であることがわかる。
\end{proof}



\bfsubsection{補題 1}
\barquo{
$a \notin R_i$なら任意の$s \geq 2$が条件をみたす。
}
\begin{proof}
  $a \notin R_i$なら任意の$s \geq 2$について$(1 + a + \cdots + a^{s-1})^{-1} \in R_i$かつ$a \cdot (1 + a + \cdots + a^{s-1})^{-1}$を満たすことを示そう。

  まず$s=2$のときを考える。$a \notin R_i$より$1 + a \notin R_i$である。したがって$(1+a)^{-1} \in R_i$がわかる。またこのとき$a(1+a)^{-1} = 1 - (1+a)^{-1} \in R_i$である。

$s \geq 3$のとき、$a^{2-s}(1 + a + \cdots + a^{s-1}) = (a^{2-s} + a^{3-s} +  \cdots + 1) + a \notin R_i$なので、$1 + a + \cdots + a^{s-1} \not\in R_i$である。とくに$(1 + a + \cdots + a^{s-1})^{-1} \in R_i$がわかる。
また$a^{-1}(1 + a + \cdots + a^{s-1}) = a^{-1} + (1 + a + \cdots + a^{s-2}) \not\in R_i$だから、$a(1 + a + \cdots + a^{s-1})^{-1} \in R_i$である。
  %$s \geq 2$とし、$s$以下の場合は示されたとする。このとき、$1 + a + \cdots + a^{s}  = 1 + a(1 + a + \cdots + a^{s-1})$である。ところが$a^{-1}(1 + a + \cdots + a^{s-1})^{-1} \in \frakm_i$なので、$a(1 + a + \cdots + a^{s-1}) \notin R_i$である。
  %ゆえに$1 + a + \cdots + a^{s} \notin R_i$で、したがって$(1 + a + \cdots + a^{s})^{-1} \in R_i$である。また
  %\begin{align*}
  %  a \cdot (1 + a + \cdots + a^{s})^{-1} &= \f{a}{ 1 + a + \cdots + a^{s} } \\
  %  &= \f{ a( 1 + a + \cdots + a^{s-1}) }{  1 + a + \cdots + a^{s} } \f{1}{ 1 + a + \cdots + a^{s-1}} \\
  %  &= \left( 1 - \f{1}{ 1 + a + \cdots + a^{s}} \right) \f{1}{ 1 + a + \cdots + a^{s-1}}
  %\end{align*}
  %であるから、$a \cdot (1 + a + \cdots + a^{s})^{-1} \in R_i$がわかる。
\end{proof}



\bfsubsection{補題 1}
\barquo{
もし$1-a \equiv 0 \; (\frakm_i)$なら、$s$が$R_i / \frakm_i$の標数の倍数でなければよい。
}
\begin{proof}
  対偶を示す。$p \geq 0$を$R_i / \frakm_i$の標数とする。$(1 + a + \cdots + a^{s-1})^{-1} \notin R_i$と仮定して$s \in p\Z$を示そう。このとき$1 + a + \cdots + a^{s-1} \in \frakm_i$なので、$0 \equiv 1 + a + \cdots + a^{s-1} \equiv s \; \mod \frakm_i$である。
  よって$s \in p\Z$が結論される。
\end{proof}


\bfsubsection{補題 1}
\barquo{
$1-a \not\equiv 0 \; (\frakm_i)$だが$1 - a^t \equiv 0 \; (\frakm_i)$となる$t \geq 2$が存在するときには、そのような最小の$t$を$t_0$とすれば、$1-a^s \equiv 0 \; (\frakm_i)$となるのは$s$が$t_0$の倍数のときであるからそれを避ければよい。
}
\begin{proof}
  $\exists t \geq 2  \quad 1 - a^t \equiv 0  \mod \frakm_i$が成り立つので、$a \in (R_i/\frakm_i)^{\tm}$である。このとき$t_0$は$a \in (R_i/\frakm_i)^{\tm}$の位数に一致する。したがって結論が従う。
\end{proof}


\bfsubsection{定理 12.2}
\barquo{
$A$の極大イデアル$I$がどの$\frakp_i$にも含まれないとすると、$I$の元$x$で$\bigcup_{i=1}^n \frakp_i$に入らないものが存在する。
}
\begin{rem}
  次の有名な補題(演習問題1.6)に帰着する。
  \lem{
  (Prime Avoidance Lemma)\\
  $R$は環、$I \subset R$と$\frakp_1 , \cdots , \frakp_n \subset R$はイデアルだとし、$3 \leq i \leq n$なら$P_i$は素イデアルであるとする。このとき、すべての$i$について$I \not\subset \frakp_i$ならば、$I \not\subset \bigcup_{i=1}^n \frakp_i$である。
  }
  \begin{proof}
    $n$についての帰納法で示す。$n=1$のときはあきらか。$n \geq 1$とし、$n$以下までは成立するとしよう。仮定により
\[
z_i \in I \sm \bigcup_{j \neq i} \frakp_j
\]
なる$z_i \; (1 \leq i \leq n+1)$がある。もしも$\exists i \; z_i \not\in \frakp_i$ならば示すべきことはないので、$\forall i \; z_i \in \frakp_i$としてよい。ここで
\[
z = \prod_{i=1}^n z_i + z_{n+1}
\]
とする。$n \geq 2$なら$\frakp_{n+1}$は素イデアルなので、このとき$\prod_{i=1}^n z_i \not\in \frakp_{n+1}$である。($n =1$のときはあきらか)ゆえに$z \in I \sm \bigcup_{i=1}^{n+1} z_{i}$なので、示すべきことがいえた。
  \end{proof}
\end{rem}



\bfsubsection{定理 12.2}
\barquo{
各$R_i$がDVRなら$\frakm_i \neq \frakm_i^2$, よって$\frakp_i \neq \frakp_i^{(2)}$である
}
\begin{proof}
  $\frakp_i = \frakp_i^{(2)}$と仮定して$\frakm_i = \frakm_i^2$を示せばよい。それは
  \begin{align*}
    \frakm_i &= \frakp_i A_{\frakp_i} \\
    &= \frakp_i^{(2)} A_{\frakp_i} \\
    &= (\frakp_i^2 A_{\frakp_i} \cap A) A_{\frakp_i} \\
    &= \frakp_i^2 A_{\frakp_i} \\
    &= \frakm_i^2
  \end{align*}
  から従う。
  \begin{comment}
  $R = R_i$, $\frakp = \frakp_i$, $\frakm = \frakm_i$として添え字を省略する。背理法で示す。$\frakp = \frakp^{(2)}$とする。$R$はDVRなので、$\frakm = wR$なる$w \in R$がある。よって$R = A_{\frakp}$により、$sw \in A$なる$s \in A \sm \frakp$がある。
  このとき$sw \in \frakm \cap A = \frakp$なので$sw \in \frakp_i^{(2)}$である。ここで$\frakp_i^{(2)} = \frakp^2 A_{\frakp} \cap A = \frakm^2 \cap A$であることから、$sw = w^2t$なる$t \in R$がある。このとき$s = wt$より、$s \in \frakm \cap A = \frakp$となって矛盾。
\end{comment}
\end{proof}


\bfsubsection{定理 12.2}
\barquo{
したがって$\frakp_i$の元$x_i$で$\frakp_i^{(2)}$にも入らず、どの$\frakp_j \; (j \neq i)$にも入らないものが存在する。すると$\frakp_i = x_i A$である。
}
\begin{proof}
  $x_i$の存在はPrime Avoidance Lemmaから。$A$の極大イデアルは$\frakp_j$だけなので、任意の$j$について
  \[
  \frakp_i / x_i A \ts_A A_{\frakp_j} = 0
  \]
  であることを見ればよい。局所化の平坦性と$x_i$の定め方より$j \neq i$のときは、あきらかに
  \begin{align*}
    \frakp_i / x_i A \ts_A A_{\frakp_j} &= \Coker(x_i A \to \frakp_i) \ts A_{\frakp_j} \\
    &= \Coker(x_i A \ts A_{\frakp_j} \to \frakp_i \ts A_{\frakp_j})  \\
    &= \Coker(x_i A \ts A_{\frakp_j} \to \frakp_i \ts A_{\frakp_j} \to \frakp_i A_{\frakp_j}) \\
    &= \frakp_i A_{\frakp_j} / x_i A_{\frakp_j} \\
    &= 0
  \end{align*}
である。そこで$i=j$とする。各$R_{i}$はDVRなので、$x_i A_{\frakp_i} = \frakm_i^{r}$なる$r \geq 1$がある。$x_i \not\in \frakp_i^{(2)}$なので、$r=1$でなくてはならない。ゆえに$\frakp_i / x_i A \ts_A A_{\frakp_i} = 0$もいえる。よって定理4.6より、$\frakp_i = x_i A $である。
\end{proof}



\bfsubsection{定理 12.2}
\barquo{
$I$を$A$の任意のイデアルとし$I R_i = x_i^{v_i} R_i$とおけば$I = x_1^{v_1} \cdots x_n^{v_n} A$であることが容易にわかる。
}
\begin{proof}
  $z = x_1^{v_1} \cdots x_n^{v_n}$とする。このとき$zR_i = x_i^{v_i} R_i = I R_i$だから、$(zA+I)R_i = I R_i$である。とくにすべての$i$について、局所化の平坦性から
  \[
  (zA + I)/I \ts_A A_{\frakp_i} = 0
  \]
  だから、$zA + I = I$であり、とくに$zA \subset I$である。
  したがって$I/zA$を考えることができる。$I/zA$もまた$\frakp_i$での局所化がすべて消えているため、$I = zA$がいえる。
\end{proof}



\bfsubsection{定理 12.3}
\barquo{
もし$\frakm_{\grl} \cap A = (0)$なら$R_{\grl} \supset K$となり矛盾するから
}
\begin{proof}
$\frakm_{\grl} \cap A = (0)$とする。このとき$A \sm \{ 0 \} \subset R_{\grl} \sm \frakm_{\grl}$である。いま$x \in K$が与えられたとする。$K$は$A$の商体なので、$ax \in A $なる$a \in A \sm \{ 0 \}$がある。
$R_{\grl}$は局所環だからこのとき$a \in R_{\grl}^{\tm}$であり、したがって$x \in R_{\grl}$がわかる。
\end{proof}



\bfsubsection{定理 12.3}
\barquo{
$A \supset \bigcap_{\mathrm{ht} \frakp = 1} A_{\frakp}$を示せばよい。すなわち
\[
a,b \in A, \; a \neq 0, \; b \in a A_{\frakp} \; (\forall A_{\frakp} \in \scrs_0) \; \To \; b \in aA
\]
をいえばよい。容易にわかるように、これは$aA$が高度$1$の準素イデアルの共通部分として表せることと同値である。
}
\begin{proof}
  まず次の補題を示す。
  \lem{
  $A$は体でないKrull環、$\frakp \subset A$を高さ$1$の素イデアルとする。$\frakq \subset A$はイデアルとする。このとき次は同値。
  \begin{description}
    \item[(1)] $\frakq$は$\frakp$に属する準素イデアル。
    \item[(2)] $\frakq$は$\frakp$の記号的$n$乗$\frakp^{(n)} = \frakp^n A_{\frakp} \cap A$のいずれかと一致する。
  \end{description}
  }
  \begin{proof} ${}$
    \begin{description}
      \item[(1)$\To$(2)] $A$の$\frakp$に属する準素イデアルと、$A_{\frakp}$の$\frakp A_{\frakp}$に属する準素イデアルとの間には自然な全単射がある。したがって$\frakq = \frakq A_{\frakp} \cap A$である。ここで$A$はKrull環で$\height \frakp = 1$なので、$A_{\frakp}$はDVRである。
      よって$\frakq A_{\frakp}= \frakp^n A_{\frakp}$なる$n \geq 1$がある。
      \item[(2)$\To$(1)] $\frakp^n A_{\frakp} \subset A_{\frakp}$は、$A_{\frakp}$がDVRなので、あきらかに$\frakp A_{\frakp}$に属する準素イデアル。よって、準同形$A \to A_{\frakp}$で引き戻すことにより、すべての$n$について$\frakp^{(n)}$は$\frakp$に属する準素イデアルであるとわかる。とくに$\frakq$もそうである。
    \end{description}
  \end{proof}
  引用部の証明に戻る。『同値である』と書いてあるが、使うのは一方向だけなので、そちらだけ示す。$a,b \in A, \; a \neq 0, \; b \in a A_{\frakp} \; (\forall A_{\frakp} \in \scrs_0)$と仮定する。$aA$が高度$1$の準素イデアルの共通部分として$aA = \frakq_1 \cap \cdots \cap \frakq_r$と表せたとする。各$\frakq_i$は素イデアル$\frakp_i$に属しているとしよう。
  このとき$\frakq_i$は記号べきとして$\frakq_i = \frakp_i^{(n_i)}$と表せる。
  $\frakp_i$は高さ$1$なので、$A_{\frakp_i}$はDVRである。$A_{\frakp_i}$の付値を$v_i$とすれば
  \[
\frakp_i^{(n_i)} = \setmid{x \in A}{ v_i(x) \geq n_i }
  \]
  だから、結局
  \[
  aA = \setmid{x \in A}{ \forall i \; v_i(x) \geq n_i}
  \]
  がわかる。

  いま、$b$についての仮定からとくに$b \in \bigcap_{i} a A_{\frakp_i}$である。したがって任意の$i$について$b \in a A_{\frakp_i} \cap A = \frakp_i^{(v_i(a))} = \setmid{x \in A}{v_i(x) \geq v_i(a)}$である。$v_i(a) \geq n_i$なので、$b \in aA$がわかる。
\end{proof}


\bfsubsection{定理 12.3}
\barquo{
\[
aR_i \cap A = \frakq_i, \quad \rad (R_i) \cap A = \frakp_i
\]
とおけば$\frakq_i$は$\frakp_i$に属する準素イデアルであり$aA = \frakq_1 \cap \cdots \cap \frakq_t$である。
}
\begin{proof}
  準素イデアルは準同形による引き戻しで保たれる。また、準素イデアルを準同形で引き戻すと、それに付随して上にある素イデアルも引き戻される。より正確には次が成り立つ。
  \lem{
  (準素イデアルの引き戻し) \\
  $\phi \colon A \to B$が環の準同形とする。$\frakq \subset B$が準素イデアルであり、$\sqrt{\frakq} = \frakp$ならば、$\phi^{-1}(\frakq) \subset A$は準素イデアルであり、$\sqrt{\phi^{-1}(\frakq)} = \phi^{-1}(\frakp)$が成り立つ。
  }
  \begin{proof}
    あきらかだが、証明をみたければ雪江\cite{雪江3} 命題2.1.7を参照のこと。
  \end{proof}
  引用部の証明に戻る。$R_i$はDVRなので、$aR_i$は$\rad (R_i)$に属する準素イデアルとなる。また$aA \subset \frakq_1 \cap \cdots \cap \cap \frakq_t$はあきらか。逆は
  \begin{align*}
  \frakq_1 \cap \cdots \cap \cap \frakq_t &= \bigcap_{\grl \in \grL} a R_{\grl} \cap A \\
  &\subset a \bigcap_{\grl \in \grL} R_{\grl} \\
  &= aA
  \end{align*}
  と示せる。
\end{proof}


\bfsubsection{定理 12.4}
\barquo{
i) ネータ正規整域はKrull環である。
}
\begin{proof}
  ネータ正規整域$A$が与えられたとする。$K = \Frac A$とする。定理11.5により、$A = \bigcap_{\height P = 1} A_P$である。各$A_P$はあきらかに$1$次元ネーター局所整閉整域なのでDVRである。あとは、各$a \in K^{\tm}$に対しゼロでない付値をあたえる$P$が有限個であることを示せばよい。$K$は$A$の商体なので、$a \in A \sm \{0\}$として示せば十分である。よって$a \in P A_P \cap A$、つまり$a \in P$なる高度$1$の素イデアルが有限個であることを示せばよいことになる。

  $A$加群$A/aA$を考える。このとき素イデアル$P$に対して
  \begin{align*}
    P \in \Supp(A/aA) &\iff A/aA \ts_A A_P \neq 0 \\
    &\iff A_P / a A_P \neq 0 \\
    &\iff aA \subset P
  \end{align*}
  である。とくに$\height P=1$なら、$aA \subset P$は、$P$が$\Supp(A/aA)$の極小元であることを意味する。($A$は整域であることも使った) よって、定理6.5により$a \in P$なる高度$1$の素イデアルは$\Ass(A/aA)$の元である。ここで$A/aA$は有限生成$A$加群で、$A$はNoetherなので$\Ass(A/aA)$は有限集合。したがって、示すべきことがいえた。
\end{proof}


\bfsubsection{定理 12.4}
\barquo{
ii) $A$を整域、$K$を$A$の商体、$L$を$K$の拡大体とする。$L$に含まれるKrull環の族$\{ A_i \}_{i \in I}$があって、(1) $A = \bigcap A_i$, (2) $0 \neq a \in A$なら有限個の$i$を除いて$a A_i = A_i$, の2条件が成り立つならば、$A$はKrull環である。
}
\begin{proof} ${}$
  \begin{description}
    \item[Step 1] $A_i$にKrull環の構造を定めるDVR族$\{ R_{\grl} \}_{\grl \in \grL_i}$をとる。$R_{\grl}$の極大イデアルを$\frakm_{\grl}$とおく。あきらかに$K \subset \Frac A_i$なので、
    \[
\grL'_i = \setmid{\grl \in \grL_i}{K \cap \frakm_{\grl} \neq (0)}
    \]
    とすれば$A_i \cap K = \bigcap_{i \in \grL'_i} K \cap R_{\grl}$であり、これによって$A_i \cap K$はKrull環となる。ここで$\Frac (R_{\grl} \cap K) = K$であることに注意する。このとき
    \[
    A = \bigcap_{i \in I} \bigcap_{\grl \in \grL'_i} K \cap R_{\grl}
    \]
    である。これが$A$にKrull環の構造を定めることを示そう。
    \item[Step 2] $a \in A \sm \{0\}$が与えられたとする。$A$の商体が$K$なので、
    \[
    J = \setmid{ (i,\grl) \in \coprod_{i \in I} \grL'_i }{a \in K \cap \frakm_{\grl}}
    \]
    として、$\# J < \infty$を示せば十分である。いま
    \begin{align*}
      J_A &= \setmid{i \in I}{a  \notin A_i^{\tm} } \\
      J_i &= \setmid{ \grl \in \grL'_i}{a \in K \cap \frakm_{\grl}}
    \end{align*}
    とする。条件(2)より$\# J_A < \infty$であり、各$A_i \cap K$はKrull環であることにより$\# J_i < \infty$である。したがって
    \[
    J \subset \coprod_{i \in J_A} J_i
    \]
    より$\# J < \infty$がわかる。以上により、$A$はKrull環である。
  \end{description}
\end{proof}



\bfsubsection{定理 12.4}
\barquo{
iii) $A$がKrull環ならば$A[X]$, $A[[X]]$もそうである。
}
\begin{proof} ${}$
  \begin{description}
\item[Step 1] $K = \Frac A$とする。$K[X]$はPIDなのでKrull環である。先に進むために次の補題を用意する。
\lem{
(多項式環の局所化の簡約) \\
$A$は整域、$K$は$A$の商体、$\frakp \subset A$は素イデアルであるとする。$B = A_{\frakp}$, $\frakq = \frakp A_{\frakp}$とおく。このとき、$K[X]$の部分集合として
\begin{description}
  \item[(1)] $A[X]_{\frakp[X]} = B[X]_{\frakq[X]}$ \\
  \item[(2)] $B$がUFDならば$K[X] \cap A[X]_{\frakp[X]} = B[X]$
\end{description}
}
\begin{proof} ${}$
  \begin{description}
\item[(1)] $A \sm  \frakp \subset A[X] \sm  \frakp[X]$なので、$B[X] \subset A[X]_{\frakp[X]} $である。$f \in B[X] \sm \frakq[X]$としよう。
$f \in B[X]$より、ある$a \in A \sm \frakp$が存在して$af \in A[X]$である。また$f \notin \frakq[X]$より$af \notin \frakp[X]$である。よって
$1/f = a / (af) \in A[X]_{\frakp[X]}$である。したがって$B[X]_{\frakq[X]} \subset A[X]_{\frakp[X]}$がわかる。

逆を示そう。$A[X] \subset B[X]_{\frakq[X]}$はあきらか。$A[x] \sm \frakp[X]$の元は$B[X]_{\frakq[X]}$の単元なので$A[X]_{\frakp[X]} \subset B[X]_{\frakq[X]}$がいえる。
したがって、$A[X]_{\frakp[X]} = B[X]_{\frakq[X]}$である。
\item[(2)] ($B$がUFDという仮定は本当は不要と思われるが、証明できなかった) $B[X] \subset K[X] \cap A[X]_{\frakp[X]}$はあきらか。逆に$f \in K[X] \cap A[X]_{\frakp[X]}$が与えられたとする。
(1)より$f \in B[X]_{\frakq[X]}$であり、
このとき$f \in A[X]_{\frakp[X]}$より$f = g/h$なる$h \in B[X] \sm \frakq[X]$と$g \in B[X]$がある。$B$は局所環なので、$h$は原始多項式である。
$B[X]$はUFDなので、次のGaussの補題の系が使える。
\lem{
(Gaussの補題の系) \\
$B$を一意分解環、$B$の商体を$K$、$g, h \in B[X]$で$h$は原始多項式とする。このとき、$g/h \in K[X]$ならば実は$g/h \in B[X]$である。
}
\begin{proof}
  証明は雪江\cite{雪江2} 補題1.11.33を参照のこと。
\end{proof}
簡約補題の証明に戻る。したがって、$f \in K[X]$より$f \in B[X]$がいえる。これで示すべき事がいえた。

  \end{description}
\end{proof}
  \end{description}
\end{proof}


\begin{thebibliography}{1}%参考文献の リスト
  \bibitem{Osborne} M.Scott Osborne『Basic Homological Algebra』(Springer, 2000)
  \bibitem{雪江3} 雪江明彦『代数学3 代数学のひろがり』(日本評論社, 2011)
  \bibitem{雪江2} 雪江明彦『代数学2 環と体とガロア理論』(日本評論社, 2010)
  \bibitem{内田} 内田伏一『集合と位相』(裳華房, 1986)
  \bibitem{FANF} Dinakar Ramakrishnan, Robert J.Valenza『Fourier Analysis on Number Fields』(Springer, 1999)
  \bibitem{Rotman} Joseph J.Rotman『An Introduction to Homological Algebra』(Springer, 2009)
\end{thebibliography}


\end{document}
